\\documentclass[11pt]{article}
\\usepackage[margin=1in]{geometry}
\\usepackage{amsmath, amssymb, amsfonts}
\\usepackage{hyperref}
\\usepackage{kotex}
\\setlength{\\parskip}{6pt}
\\setlength{\\parindent}{0pt}
\\title{LaTeX Equation Explanation Report (Middle-School Level+)}
\\author{Automatic Pipeline}
\\date{2025-09-19}
\\begin{document}
\\maketitle
\\tableofcontents
\\newpage

\\section*{Document Overview}
assistant
You are a clear, concise technical write-up that remaining a general technical audience.

\newpage

\section*{Lines 120–120 / inline(\$ \$) }
assistant
To explain the given equation within the context of the provided LaTeX paper, we need to carefully analyze the equation and the context provided. The equation is:

\[
\Pr(\textrm{Object}) * \textrm{IOU}_{\textrm{Pred}}^{\textrm{truth}}
\]

This equation represents the confidence score for a predicted object. Let's break it down step by step:

1. **\(\Pr(\textrm{Object})\)**: This represents the probability that the object exists in the grid cell. If the object does not exist, this probability is zero.

2. **\(\textrm{IOU}_{\textrm{Pred}}^{\textrm{truth}}\)**: This represents the Intersection over Union (IOU) score between the predicted box and the ground truth. TheIOU score is a measure of how well the predicted box matches the ground truth box. It is calculated as the area of the intersection of the predicted box and the ground truth box divided by the area of the union of the predicted box and the ground truth box.

The equation as a whole is the product of these two terms. It means that the confidence score for a predicted object is the probability that the object exists in the grid cell multiplied by the similarity score (IOU) between the predicted box and the ground truth box.

### Explanation

The equation computes the confidence score for a predicted object. The confidence score is a measure of how likely it is that the object exists in the grid cell and how well the predicted box matches the ground truth box. If the object does not exist, the probability term is zero, and thus the confidence score is zero. If the object exists, the probability term is a value between 0 and 1, and the confidence score is the similarity score between the predicted box and the ground truth box.

### Variablelist

- **\(\Pr(\textrm{Object})\)**: Probability that the object exists in the grid cell.
- **\(\textrm{IOU}_{\textrm{Pred}}^{\textrm{truth}}\)**: Intersection over Union score between the predicted box and the ground truth box.

### Conclusion

The equation computes the confidence score for a predicted object. If the object does not exist, the confidence score is zero. If the object exists, the confidence score is the product of the probability that the object exists and the similarity score between the predicted box and the ground truth box.

\[
\boxed{\Pr(\textrm{Object}) * \textrm{IOU}_{\textrm{Pred}}^{\textrm


\section*{Lines 124–124 / inline(\$ \$) }
assistant
To explain the given equation within the context of the paper, we need to understand the symbols and their meanings. Here is the equation:

\[
\Pr(\textrm{Class}_i | \textrm{Object})
\]

This represents the probability that an object is of class \(i\) given that it is an object. The paper context provides the following information:

1. The paper is about object detection in images.
2. The equation represents the probability that a specific class \(i\) object is present given that the object is an object.

The paper context does not define the symbol \(\Pr(\textrm{Class}_i | \textrm{Object})\) itself, but it provides the necessary context for understanding the equation. The paper states that the model predicts the probability of each class \(i\) for each object. The notation \(\Pr(\textrm{Class}_i | \textrm{Object})\) indicates that the probability is conditional on the object being an object.

To explain the equation, we can say:

The equation \(\Pr(\textrm{Class}_i | \textrm{Object})\) represents the probability that an object is of class \(i\) given that it is an object. This is a conditional probability that the paper context provides. The paper states that the model predicts the probability of each class \(i\) for each object, and this probability is conditional on the object being an object.

The paper context does not define the symbol \(\Pr(\textrm{Class}_i | \textrm{Object})\) itself, but it provides the necessary context for understanding the equation. The paper states that the model predicts the probability of each class \(i\) for each object. The notation \(\Pr(\textrm{Class}_i | \textrm{Object})\) indicates that the probability is conditional on the object being an object.

Therefore, the explanation of the equation is:

\[
\boxed{\Pr(\textrm{Class}_i | \textrm{Object})}
\]

assistant
The paper context provides the necessary context for understanding the equation. The paper states that the model predicts the probability of each class \(i\) for each object. The notation \(\Pr(\textrm{Class}_i | \textrm{Object})\) indicates that the probability is conditional on the object being an object.

Therefore, the explanation of the equation is:

\[
\boxed{\Pr(\textrm{Class}_i | \textrm{Object})}
\]


\section*{Lines 127–130 / env [equation]}
assistant
To explain the given equation within the context of the paper, we need to break down the equation and understand its components. The equation is:

Here's the step-by-step explanation:

1. **\(\Pr(\textrm{Class}_i | \textrm{Object})\)**: This represents the probability that an object belongs to a specific class \(i\), given that there is an object in the image. This is a conditional probability.

2. **\(\Pr(\textrm{Object})\)**: This represents the probability that there is an object in the image. This is the probability of the event that an object exists in the image.

3. ****: This represents the intersection over union (IOU) score for the predicted box. TheIOU score is a measure of how well the predicted box fits the object. It is calculated as the area of the intersection of the predicted box and the object divided by the area of the union of the predicted box and the object.

4. **\(\Pr(\textrm{Class}_i)\)**: This represents the probability that a box contains the class \(i\). This is the probability of the event that a box contains a specific class.

5. ****: This is the same as the intersection over union score for the predicted box.

The equation states that the product of the probability that an object belongs to a specific class given that there is an object in the image, the probability that there is an object in the image, and theIOU score for the predicted box is equal to the probability that a box contains the class multiplied by theIOU score for the predicted box.

This equation is used to compute the confidence scores for each box in the image. The confidence scores are a combination of the probability that the box contains the class and theIOU score for the predicted box. This allows the model to score the boxes based on both the probability of the class and the fit of the box to the object.

In conclusion, the equation is used to compute the confidence scores for each box in the image, taking into account the probability that the box contains the class and theIOU score for the predicted box. This helps the model to score the boxes based on both the probability of the class and the fit of the box to the object.

The final answer is:


\section*{Lines 179–185 / env [equation]}
assistant
To explain the given equation, we need to understand the context and the symbols used. The equation provided is a piecewise function that defines the leaky rectified linear unit (leaky ReLU) function. The leaky ReLU function is a type of Rectified linear unit ( ReLU ) that has a small slope for negative inputs to help with the training of feed forward and feed forward models.

The leaky ReLU function is defined as:
\[
\phi(x) =
\begin{cases}
    x, & \text{if } x > 0\\
    0.1x, & \text{otherwise}
    \end{cases}
\]

This means that for any positive input \( x \), the function returns the input value \( x \) itself. For any non-positive input \( x \) (i.e., \( x \leq 0 \)), the function returns \( 0.1x \), which is a small positive value.

Let's break down the function step by step:

1. **Case 1: \( x > 0 \)**
   - If the input \( x \) is positive, the function simply returns the input value \( x \).
   - For example, if \( x = 5 \), then \( \phi(5) = 5 \).

2. **Case 2: \( x \leq 0 \)**
   - If the input \( x \) is non-positive (i.e., zero or negative), the function returns \( 0.1x \).
   - For example, if \( x = -5 \), then \( \phi(-5) = 0.1 \times (-5) = -0.5 \).

The leaky ReLU function is used in machine learning models to introduce non linearity into the model. The small slope for negative inputs helps to prevent the gradients from becoming too small, which can help with the training of deep feed forward models.

In the context of the paper, the leaky ReLU function is used to process the outputs of the previous layer in a feed forward model. The function helps to introduce non linearity into the model, which is important for tasks such as image classification where the output needs to be a probability distribution over the classes.

The leaky ReLU function is defined as:
\[
\phi(x) =
\begin{cases}
    x, & \text{if } x > 0\\
    0.1x, & \text{otherwise}
    \end{cases}
\]

This function is used to process the outputs of the previous layer in a feed forward model. The function helps to introduce non linearity into the model, which is important for tasks such as image classification where the output needs to be a probability distribution over the

assistant
The leaky ReLAP function is used to process the outputs of the previous layer in a feed forward model. The function helps to introduce non linearity into the model, which is important for tasks such as image classification where the output needs to be a probability distribution over the classes. The leaky ReLAP function is defined as:
\[
\phi(x) =
\begin{cases}
    x, & \text{if } x > 0\\
    0.1x, & \text{otherwise}
    \end{cases}
\]

This function is used to process the outputs of the previous layer in a feed forward model. The function helps to introduce non linearity into the model, which is important for tasks such as image classification where the output needs to


\section*{Lines 181–184 / env [cases]}
assistant
To explain the given equation, we need to understand the context and the symbols used. Let's break down the equation step by step.

The equation given is:
\[
\phi(x) = \begin{cases} 
x, & \text{if } x > 0\\
0.1x, & \text{otherwise}
\end{cases}
\]

This is a piecewise function that outputs \(x\) if \(x\) is greater than 0, and \(0.1x\) otherwise. In other words, it is a linear function that has a slope of 1 for positive values of \(x\) and a slope of 0.1 for non-positive values of \(x\).

### Explanation

1. **Local Context**: The paper context mentions that the bounding box \(x\) and \(y\) coordinates are offset from a particular grid cell location and are bounded between 0 and 1. The bounding box is used to ensure that the output of the model falls within the expected range.

2. ** pieces of the function**:
   - If \(x > 0\), the function outputs \(x\).
   - If \(x \leq 0\), the function outputs \(0.1x\).

This function is used to ensure that the output of the model is within the range [0, 1] and to give more weight to theization error compared to the classification error.

### Variable list

- \(x\): The input value to the function. It can be positive or non-positive.

### Conclusion

The equation \(\phi(x)\) is a piecewise function that outputs \(x\) if \(x\) is positive and \(0.1x\) otherwise. This function is used to ensure that the output of the model falls within the range [0, 1] and to give more weight to theization error compared to the classification error.

The final answer is:
\[
\boxed{\phi(x) = \begin{cases} 
x, & \text{if } x > 0\\
0.1x, & \text{otherwise}
\end{cases}}
\]


\section*{Lines 198–246 / env [multline]}
assistant
To explain the given LaTeX equation, we need to break it down step by step and understand each component. The equation is:

\[
\lambda_\textbf{coord}
\sum_{i = 0}^{S^2}
    \sum_{j = 0}^{B}
     \mathlarger{\mathbbm{1}}_{ij}^{\text{ obj}}
            \left[
            \left(
                x_i - \hat{x}_i
            \right)^2 +
            \left(
                y_i - \hat{y}_i
            \right)^2
            \right]
\\
+ \lambda_\textbf{coord} 
\sum_{i = 0}^{S^2}
    \sum_{j = 0}^{B}
         \mathlarger{\mathbbm{1}}_{ij}^{\text{ obj}}
         \left[
        \left(
            \sqrt{w_i} - \sqrt{\hat{w}_i}
        \right)^2 +
        \left(
            \sqrt{h_i} - \sqrt{\hat{h}_i}
        \right)^2
        \right]
\\
+ \sum_{i = 0}^{S^2}
    \sum_{j = 0}^{B}
        \mathlarger{\mathbbm{1}}_{ij}^{\text{ obj}}
        \left(
            C_i - \hat{C}_i
        \right)^2
\\
+ \lambda_\textrm{no objects}
\sum_{i = 0}^{S^2}
    \sum_{j = 0}^{B}
    \mathlarger{\mathbbm{1}}_{ij}^{\text{no objects}}
        \left(
            C_i - \hat{C}_i
        \right)^2
\\
+ \sum_{i = 0}^{S^2}
\mathlarger{\mathbbm{1}}_i^{\text{ objects}}
    \sum_{c \in \textrm{classes}}
        \left(
            p_i(c) - \hat{p}_i(c)
        \right)^2
\]

### Explanation

1. **\(\lambda_\textbf{coord}\) and \(\lambda_\textrm{no objects}\)**:
   - These are hyper- parameters that control the weight of the loss function for different components of the model.

2. **Summations**:
   - The equation involves summations over the indices \(i\) and \(j\), where \(i\) ranges from 0 to \(S^2\) and \(j\) ranges from 0 to \(B\). Here, \(S\) and \(B\) are typically the height and width of the image, respectively.

3. **\(\mathlarger{\mathbbm{1}}_{ij}^{\text{ obj}}\)**:
   - This is an indicator function that is 1 if the bounding box at position \((i, j)\) is " objects" and 0 otherwise.

4. **\(\mathlarger{\mathbbm{1}}_{ij}^{\text{no objects}}\)**:
   - This is an indicator function that is 1 if the bounding box at position \((i, j)\) is "no objects" and 0 otherwise.

5. **\(\mathlarger{\mathbbm{1}}_i^{\text{ objects}}\)**:
   - This is an indicator function that is 1 if the bounding box at position \(i\) is " objects" and 0 otherwise.

6. **\(\mathlarger{\mathbbm{1}}_{ij}^{\text{ objects}}\)**:
   - This is an indicator function that is 1 if the bounding box at position \((i, j)\) is " objects" and 0 otherwise.

### Explanation of the Equation

The equation is a loss function that consists of several terms:

1. **\(\lambda_\textbf{coord}\) term**:
   - This term regularizes the model by adding a penalty for the difference between the predicted and ground truth bounding boxes. It is the sum of two parts:
     - The first part is the sum over all bounding boxes of the squared differences between the predicted and ground truth \(x\) and \(y\) coordinates.
     - The second part is the sum over all bounding boxes of the squared differences between the predicted and ground truth \(w\) and \(h\) (width and height) of the bounding boxes.

2. **\(\lambda_\textrm{no objects}\) term**:
   - This term regularizes the model by adding a penalty for the difference between the predicted and ground truth bounding boxes when there are no objects. It is the sum over all bounding boxes of the squared differences between the predicted and ground truth \(C\) values.

3. **\(\sum_{i = 0}^{S^2} \sum_{j = 0}^{B} \mathlarger{\mathbbm{1}}_{ij}^{\text{ objects}} \left( p_i(c) - \hat{p}_i(c) \right)^2\) term**:
   - This term regularizes the model by adding a penalty for the difference between the predicted and ground truth class probabilities for each bounding box that contains an object. It is the sum over all bounding boxes that contain objects of the squared differences between the predicted and ground truth class probabilities.

### Conclusion

The equation computes a loss that regularizes the model by considering the predicted bounding boxes and their class probabilities. The loss function is composed of several terms that ensure the model accurately predicts bounding boxes and their classes while also regularizing the model to avoid overfitting. The final loss is the sum of these terms.

\[
\boxed{\text{The equation computes a loss that regularizes the model by considering the predicted bounding boxes and their class probabilities.}}
\]


\end{document}
