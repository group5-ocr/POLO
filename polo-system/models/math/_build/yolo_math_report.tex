\\documentclass[11pt]{article}
\\usepackage[margin=1in]{geometry}
\\usepackage{amsmath, amssymb, amsfonts}
\\usepackage{hyperref}
\\usepackage{kotex}
\\setlength{\\parskip}{6pt}
\\setlength{\\parindent}{0pt}
\\title{LaTeX Equation Explanation Report (Middle-School Level+)}
\\author{Automatic Pipeline}
\\date{2025-09-18}
\\begin{document}
\\maketitle
\\tableofcontents
\\newpage

\\section*{Document Overview}
assistant
You are a clear, concise technical write-up that remaining the same as the original write-up.

\newpage

\section*{Lines 102–106 / env [align]}
assistant
### Explanation

The given equations are:

\[
\sum_{n=1}^N\mathbb{E}(X_n^2) \leq N\cdot P 
\]
\[
\sum_{n=1}^N\mathbb{E}(\overline{X}_n^2) \leq N\cdot \overline{P}
\]

Let's break down each part of the equations step by step.

1. **Understanding the Summation Notation:**
   - The symbol \(\sum\) represents the sum of a sequence.
   - The index \(n\) starts at 1 and goes up to \(N\).
   - \(\mathbb{E}(X_n^2)\) represents the expected value of the square of the random variable \(X_n\).
   - \(\overline{X}_n\) represents the sample mean of the random variable \(X_n\).
   - \(\overline{P}\) represents the sample proportion of the random variable \(P\).

2. **Interpreting the Inequalities:**
   - The first inequality states that the sum of the expected values of the squares of \(N\) random variables \(X_n\) is less than or equal to \(N\) times the expected value of the square of a constant \(P\).
   - The second inequality states that the sum of the expected values of the squares of the sample means of \(N\) random variables \(\overline{X}_n\) is less than or equal to \(N\) times the expected value of the square of a sample proportion \(\overline{P}\).

3. **Conclusion:**
   - Both inequalities are essentially stating that the sum of the expected values of the squares of the random variables (or their sample means) is bounded by a linear function of the expected value of the square of a constant or a sample proportion.

### Final Answer

\[
\boxed{\sum_{n=1}^N\mathbb{E}(X_n^2) \leq N\cdot P, \quad \sum_{n=1}^N\mathbb{E}(\overline{X}_n^2) \leq N\cdot \overline{P}}
\]

assistant
The final answer is \(\boxed{\sum_{n=1}^N\mathbb{E}(X_n^2) \leq N\cdot P, \quad \sum_{n=1}^N\mathbb{E}(\overline{X}_n^2) \leq N\cdot \overline{P}}\).


\end{document}
