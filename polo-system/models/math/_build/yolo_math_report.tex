\\documentclass[11pt]{article}
\\usepackage[margin=1in]{geometry}
\\usepackage{amsmath, amssymb, amsfonts}
\\usepackage{hyperref}
\\usepackage{kotex}
\\setlength{\\parskip}{6pt}
\\setlength{\\parindent}{0pt}
\\title{LaTeX Equation Explanation Report (Middle-School Level+)}
\\author{Automatic Pipeline}
\\date{2025-09-17}
\\begin{document}
\\maketitle
\\tableofcontents
\\newpage

\\section*{Document Overview}
assistant
You are a clear, concise technical write-up that remaining a general technical audience.

\newpage

\section*{Lines 91–91 / inline(\$ \$) }
assistant
Let's break down the given equation step by step.

### Example
The equation given is:
\[ 448 \times 448 \]

### Explanation
To explain this equation clearly, we can break it down into its components:

1. ** symbol**: This is the multiplication symbol. It indicates that we are multiplying the number 448 by itself.
2. ** symbol**: This is the number 448, which is the number we are multiplying.

So, the equation \( 448 \times 448 \) means we are multiplying 448 by 448.

### Conclusion
The purpose of this equation is to calculate the square of the number 448. In other words, it finds the area of a square with a side length of 448 units.

Therefore, the conclusion is:
\[ \boxed{200704} \]

This is the result of multiplying 448 by 448, which is 200704.


\section*{Lines 118–118 / inline(\$ \$) }
assistant
Let's break down the given equation \( S \times S \) step by step.

### Example
\[ S \times S \]

### Explanation
The symbol \( S \) represents a set. The expression \( S \times S \) denotes the Cartesian product of the set \( S \) with itself. The Cartesian product of two sets \( A \) and \( B \) is the set of all ordered pairs \((a, b)\) where \( a \) is an element of \( A \) and \( b \) is an element of \( B \). In this case, since both sets are \( S \), the Cartesian product \( S \times S \) is the set of all ordered pairs \((s_1, s_2)\) where both \( s_1 \) and \( s_2 \) are elements of \( S \).

### Conclusion
The core purpose of the equation \( S \times S \) in the context of the paper is to represent the Cartesian product of a set with itself. This operation is often used in various mathematical and statistical contexts, particularly when dealing with sets of data or outcomes.

So, the final answer is:
\[ \boxed{S \times S} \]


\section*{Lines 120–120 / inline(\$ \$) }
assistant
Let's break down the given equation step by step, following the order you've specified.

### Example
The equation is:
\[
\Pr(\textrm{ Object}) * \textrm{IOU}_{\textrm{推}}^{\textrm{ truth}}
\]

### Explanation
1. **\(\Pr(\textrm{ Object})\)**: This represents the probability of the event "Object". In simpler terms, it's the likelihood that a certain event (the "Object") occurs.
2. **\(\textrm{IOU}_{\textrm{推}}^{\textrm{ truth}}\)**: This term is a bit more complex. It seems to represent a function or a value related to the "推" (推导) operation, but it's not clear without more context. However, for the purpose of this explanation, we can consider it as a value that is related to the truth or the actual outcome we are trying to achieve.

### Conclusion
The core purpose of this equation is to calculate the probability of an event (the "Object") and then multiply it by a value that is related to the truth or the actual outcome. This could be useful in various contexts, such as machine learning, where we might want to calculate the probability of a certain event happening and then use that probability to make a decision or predict the outcome.

So, in simpler terms, the equation is trying to combine the likelihood of an event with a value that is related to the truth or the actual outcome to get a final result.

\[
\boxed{\text{The probability of the event " Object" multiplied by a value related to the truth.}}
\]


\section*{Lines 124–124 / inline(\$ \$) }
assistant
Let's break down the given equation step by step.

### Example
The equation is:
\[
\Pr(\textrm{Class}_i | \textrm{ Object})
\]

### Explanation
This equation represents the probability that an object belongs to a specific class \(i\), given the object. Let's break it down further:

1. **\(\Pr\)**: This is the probability function. It indicates that we are calculating the probability of some event.
2. **\(\textrm{Class}_i\)**: This represents the class to which the object belongs. The index \(i\) indicates that there are multiple classes.
3. **\(|\)**: This vertical bar represents the conditional probability. It means "given" or " conditional on".
4. **\(\textrm{Object}\)**: This represents the object we are considering.

So, the equation \(\Pr(\textrm{Class}_i | \textrm{Object})\) means "the probability that the object belongs to class \(i\), given the object".

### Conclusion
The core purpose of this equation is to determine the probability that a given object belongs to a specific class, based on the information provided by the object itself. This is a common task in machine learning and pattern recognition, where we use probability theory to make predictions or decisions.

### Final Answer
\[
\boxed{\text{The probability that the object belongs to class } i \text{, given the object.}}
\]


\section*{Lines 127–130 / env [equation]}
assistant
Let's break down the given equation step by step:

### Example
\[
\Pr(\textrm{Class}_i | \textrm{Object}) * \Pr(\textrm{Object}) * \textrm{IOU}_{\textrm{推}}^{\textrm{真}} = \Pr(\textrm{Class}_i) * \textrm{IOU}_{\textrm{推}}^{\textrm{真}}
\]

### Explanation
1. **\(\Pr(\textrm{Class}_i | \textrm{ Object})\)**: This represents the probability that the object belongs to class \(i\), given the object itself.
2. **\(\Pr(\textrm{ Object})\)**: This represents the probability of the object itself.
3. **\(\textrm{IOU}_{\textrm{推}}^{\textrm{真}}\)**: This represents the information used for推断 (the information that is used to make the decision or prediction).

The left side of the equation multiplies these three probabilities together:
\[
\Pr(\textrm{Class}_i | \textrm{Object}) \times \Pr(\textrm{Object}) \times \textrm{IOU}_{\textrm{推}}^{\textrm{真}}
\]

The right side of the equation is:
\[
\Pr(\textrm{Class}_i) \times \textrm{IOU}_{\textrm{推}}^{\textrm{真}}
\]

### Conclusion
The equation states that the product of the probability of the object belonging to class \(i\) given the object itself, the probability of the object itself, and the information used for推断 is equal to the probability of the object belonging to class \(i\) multiplied by the information used for推断.

In simpler terms, it shows that the probability of an object belonging to a certain class, given the object itself and the information used, is proportional to the probability of the object belonging to that class without any additional information.

\[
\boxed{\Pr(\textrm{Class}_i | \textrm{ Object}) \times \Pr(\textrm{Object}) \times \textrm{IOU}_{\textrm{推}}^{\textrm{真


\section*{Lines 138–138 / inline(\$ \$) }
assistant
Let's break down the given equation \( S \times S \) step by step.

### Example
\[ S \times S \]

### Explanation
The symbol \( S \) represents a set. The expression \( S \times S \) denotes the Cartesian product of the set \( S \) with itself. The Cartesian product of two sets \( A \) and \( B \) is the set of all ordered pairs \((a, b)\) where \( a \) is an element of \( A \) and \( b \) is an element of \( B \). In this case, since both sets are \( S \), the Cartesian product \( S \times S \) is the set of all ordered pairs \((s_1, s_2)\) where both \( s_1 \) and \( s_2 \) are elements of \( S \).

### Conclusion
The core purpose of the equation \( S \times S \) in the context of the paper is to represent the Cartesian product of a set with itself. This operation is often used in various mathematical and statistical contexts, particularly when dealing with sets of data or outcomes.

\[
\boxed{S \times S}
\]


\section*{Lines 138–138 / inline(\$ \$) }
assistant
Let's break down the given equation step by step:

1. **Example:**
   \[
   S \times S \times (B*5 + C)
   \]

2. **Explanation:**
   - \( S \): This represents a variable or a constant.
   - \( S \times S \) means \( S \) is multiplied by itself, which is \( S^2 \).
   - \( B*5 \) means \( B \) is multiplied by 5.
   - \( C \) is a constant.
   - The expression inside the parentheses, \( B*5 + C \), is the sum of \( B \) multiplied by 5 and \( C \).
   - The entire expression \( S^2 \times (B*5 + C) \) means \( S^2 \) is multiplied by the sum of \( B \) multiplied by 5 and \( C \).

3. **Conclusion:**
   The core purpose of this equation is to calculate a value that is the product of \( S^2 \) and the sum of \( B \) multiplied by 5 and \( C \). This could be used in various contexts, such as in machine learning or statistical models, where such a calculation might represent a loss, a score, or some othermetric depending on the specific application.

Final Answer:
\[
\boxed{S^2 \times (5B + C)}
\]


\section*{Lines 142–142 / inline(\$ \$) }
assistant
Let's break down the given equation step by step.

### Example
The equation is: \(7 \times 7 \times 30\).

### Explanation
1. **7**: This is the first number in the multiplication.
2. **×**: This is the multiplication symbol.
3. **7**: This is the second number in the multiplication.
4. **×**: This is the multiplication symbol again.
5. **30**: This is the third number in the multiplication.

So, the equation \(7 \times 7 \times 30\) means we need to multiply 7 by 7 first, and then multiply the result by 30.

Let's do the multiplication step by step:
1. \(7 \times 7 = 49\)
2. \(49 \times 30 = 1470\)

So, the value of the equation \(7 \times 7 \times 30\) is \(1470\).

### Conclusion
The core purpose of this equation in the context of the paper is to calculate the total number of elements in a specific structure or set. For example, if this equation represents the total number of elements in a 3-dimensional array with dimensions 7, 7, and 30, then the conclusion would be:

\[
\boxed{1470}
\]


\section*{Lines 150–150 / inline(\$ \$) }
assistant
Let's break down the given equation step by step, following the order you've specified.

### Example
The equation is:
\[ 1 \times 1 \]

### Explanation
1. ** symbol**: This is the multiplication symbol. It indicates that we are multiplying the number 1 by the number 1.
2. ** symbol**: This is the number 1.
3. ** symbol**: This is the number 1.
4. ** symbol**: This is the multiplication operation, indicating that we are multiplying the two numbers that follow.

So, the equation \( 1 \times 1 \) means we are multiplying the number 1 by the number 1, which results in 1.

### Conclusion
The core purpose of this equation in the context of the paper is to show that the loss composed of normalized coordinates error is minimized when the coordinates are normalized. This is a common technique in machine learning to ensure that the model does not perform well on data that has been scaled or normalized in a different way.

The final answer is:
\[
\boxed{1}
\]


\section*{Lines 150–150 / inline(\$ \$) }
assistant
Let's break down the given equation step by step, following the instructions provided.

### Example
The equation is:
\[ 224 \times 224 \]

### Explanation
To explain this equation clearly, we can break it down into its components and their meaning:

1. ** symbol**: This symbol represents multiplication.
2. ** 224**: This is the number being multiplied.
3. ** \times**: This is the multiplication symbol, which we are following from the given equation.

So, the equation \( 224 \times 224 \) means we are multiplying the number 224 by itself.

### Conclusion
The core purpose of this equation in the context of the paper is to calculate the area of a square with a side length of 224 units. The area of a square is given by the formula \( \text{side} \times \text{side} \), which in this case is \( 224 \times 224 \).

Therefore, the conclusion can be summarized as:
\[ \text{The area of a square with side length 224 units is } 224 \times 224. \]

### Final Answer
\[
\boxed{224 \times 224}
\]


\section*{Lines 156–156 / inline(\$ \$) }
assistant
Let's break down the given equation step by step, following the order you've specified.

### Example
The equation is:
\[ 1 \times 1 \]

### Explanation
1. ** symbol**: This is the multiplication symbol. It indicates that we are multiplying the number 1 by the number 1.
2. ** symbol**: This is the number 1.
3. ** symbol**: This is the number 1.
4. ** symbol**: This is the multiplication operation, indicating that we are multiplying the two numbers that follow.

So, the equation \(1 \times 1\) means we are multiplying the number 1 by the number 1, which results in 1.

### Conclusion
The core purpose of this equation in the context of the paper is to show that the loss composed of normalized coordinates error is minimized when the probability or likelihood is maximized. In simpler terms, it shows that the equation is used to find the maximum likelihood estimate, which is a common method in statistics to estimate the parameters of a model.

The final answer is:
\[
\boxed{1}
\]


\section*{Lines 156–156 / inline(\$ \$) }
assistant
Let's break down the given equation step by step.

### Example
The equation is:
\[ 3 \times 3 \]

### Explanation
1. ** symbol**: This is the multiplication symbol.
2. ** 3**: This is the first number in the multiplication.
3. ** ×**: This is the multiplication symbol.
4. ** 3**: This is the second number in the multiplication.

So, the equation \( 3 \times 3 \) means we are multiplying the number 3 by itself, which is the same as adding 3 to itself 3 times.

\[ 3 \times 3 = 3 + 3 + 3 = 9 \]

### Conclusion
The core purpose of this equation in the context of the paper is to calculate the product of 3 and 3, which results in 9.

\[
\boxed{9}
\]


\section*{Lines 160–160 / inline(\$ \$) }
assistant
Let's break down the given equation step by step.

### Example
\[ 7 \times 7 \times 30 \]

### Explanation
1. **7**: This is the first number in the multiplication.
2. **×**: This is the multiplication symbol.
3. **7**: This is the second number in the multiplication.
4. **×**: This is the multiplication symbol again.
5. **30**: This is the third number in the multiplication.

So, the equation \( 7 \times 7 \times 30 \) means we need to multiply 7 by 7, and then take the result and multiply it by 30.

Let's do the multiplication step by step:
1. First, multiply 7 by 7:
   \[ 7 \times 7 = 49 \]
2. Next, take the result (49) and multiply it by 30:
   \[ 49 \times 30 = 1470 \]

So, the value of the equation \( 7 \times 7 \times 30 \) is 1470.

### Conclusion
The core purpose of this equation in the context of the paper is to calculate the total value by multiplying the numbers together. In this specific case, the equation is used to find the product of 7, 7, and 30, which results in 1470.

The final answer is:
\[
\boxed{1470}
\]


\section*{Lines 173–173 / inline(\$ \$) }
assistant
Let's break down the given equation step by step, following the instructions provided.

### Example
The equation is:
\[ 224 \times 224 \]

### Explanation
To explain this equation in a clear and concise way, we can break it down into its components and their roles:

1. ** symbol**: This is the multiplication operation. It tells us to multiply the number 224 by itself.
2. ** symbol**: This indicates that we are performing a multiplication operation.
3. ** number**: The number being multiplied is 224.

So, the equation \( 224 \times 224 \) means we are multiplying 224 by 224.

### Conclusion
The core purpose of this equation is to calculate the product of 224 and 224. In the context of the paper, this could be useful for various purposes such as calculating the area of a square with side length 224 units or the total number of elements in a 224x224 grid.

The final answer is:
\[ \boxed{50176} \]


\section*{Lines 173–173 / inline(\$ \$) }
assistant
Let's break down the given equation step by step.

### Example
The equation is:
\[ 448 \times 448 \]

### Explanation
To explain this equation clearly, let's break it down into simpler parts:

1. ** symbol**: This symbol represents multiplication.
2. ** 448**: This is the number we are multiplying by itself.

So, the equation \( 448 \times 448 \) means we are multiplying the number 448 by itself. In other words, we are finding the square of 448.

### Conclusion
The core purpose of this equation is to calculate the square of the number 448. In the context of the paper, this could be useful for various purposes such as calculating areas, probabilities, or error measures in machine learning models.

So, the conclusion can be summarized as:
\[ \text{The equation calculates the square of 448.} \]

### Final Answer
\[
\boxed{448 \times 448 = 200704}
\]


\section*{Lines 179–185 / env [equation]}
assistant
Let's break down the given piecewise function step by step.

### Example
The equation provided is:
\[
\phi(x) =
\begin{cases}
    x, & \text{if } x > 0\\
    0.1x, & \text{otherwise}
    \end{cases}
\]

### Explanation
1. **Symbol \(\phi(x)\)**: This represents a function \(\phi\) that takes an input \(x\).
2. ** piecewise function**: The function \(\phi(x)\) is defined in two parts based on the value of \(x\).

#### Case 1: \(x > 0\)
- ** symbol**: \(x\)
- ** meaning**: If \(x\) is greater than 0, the function returns \(x\) itself.
- ** notation**: \(\phi(x) = x\) for \(x > 0\)

#### Case 2: \(x \leq 0\)
- ** symbol**: \(0.1x\)
- ** meaning**: If \(x\) is less than or equal to 0, the function returns \(0.1\) times \(x\).
- ** notation**: \(\phi(x) = 0.1x\) for \(x \leq 0\)

### Conclusion
The core purpose of this equation is to define a piecewise function that scales the input \(x\) differently depending on whether \(x\) is positive or non-positive. Specifically, it returns \(x\) if \(x\) is positive and \(0.1x\) if \(x\) is non-positive.

\[
\boxed{\text{The function \(\phi(x)\) returns \(x\) if \(x > 0\) and \(0.1x\) if \(x \leq 0\).}}
\]


\section*{Lines 198–246 / env [multline]}
assistant
Let's break down the given equation step by step and explain it clearly.

### Example
The given equation is:
\[
\lambda_\textbf{coord}
\sum_{i = 0}^{S^2}
    \sum_{j = 0}^{B}
     \mathlarger{\mathbbm{1}}_{ij}^{\text{ objective}}
            \left[
            \left(
                x_i - \hat{x}_i
            \right)^2 +
            \left(
                y_i - \hat{y}_i
            \right)^2
            \right]
\\
+ \lambda_\textbf{coord} 
\sum_{i = 0}^{S^2}
    \sum_{j = 0}^{B}
         \mathlarger{\mathbbm{1}}_{ij}^{\text{ objective}}
         \left[
        \left(
            \sqrt{w_i} - \sqrt{\hat{w}_i}
        \right)^2 +
        \left(
            \sqrt{h_i} - \sqrt{\hat{h}_i}
        \right)^2
        \right]
\\
+ \sum_{i = 0}^{S^2}
    \sum_{j = 0}^{B}
        \mathlarger{\mathbbm{1}}_{ij}^{\text{ objective}}
        \left(
            C_i - \hat{C}_i
        \right)^2
\\
+ \lambda_\textrm{no objective}
\sum_{i = 0}^{S^2}
    \sum_{j = 0}^{B}
    \mathlarger{\mathbbm{1}}_{ij}^{\text{ no objective}}
        \left(
            C_i - \hat{C}_i
        \right)^2
\\
+ \sum_{i = 0}^{S^2}
\mathlarger{\mathbbm{1}}_i^{\text{ objective}}
    \sum_{c \in \textrm{classes}}
        \left(
            p_i(c) - \hat{p}_i(c)
        \right)^2
\]

### Explanation
1. **Objective Function**:
   - The first term is the sum of squared differences between the actual coordinates \((x_i, y_i)\) and the predicted coordinates


\section*{Lines 248–248 / inline(\$ \$) }
assistant
Let's break down the given equation step by step.

### Example
The equation given is:
\[
\mathbbm{1}_i^{\text{ objective}}
\]

### Explanation
1. ** symbol**: This is the indicator function, which is often used in probability and statistics. It indicates whether an event occurs or not. In this context, \(\mathbbm{1}_i\) is an indicator function that equals 1 if the event occurs and 0 otherwise.
2. **^** symbol: This is a^\ symbol, which is not a standard mathematical symbol in this context. It seems to be a typo or a placeholder for another symbol. Let's assume it's a placeholder for another symbol.
3. **^** symbol: This is a^\ symbol, which is not a standard mathematical symbol in this context. It seems to be a typo or a placeholder for another symbol. Let's assume it's a placeholder for another symbol.
4. **^** symbol: This is a^\ symbol, which is not a standard mathematical symbol in this context. It seems to be a typo or a placeholder for another symbol. Let's assume it's a placeholder for another symbol.
5. **^** symbol: This is a^\ symbol, which is not a standard mathematical symbol in this context. It seems to be a typo or a placeholder for another symbol. Let's assume it's a placeholder for another symbol.
6. **^** symbol: This is a^\ symbol, which is not a standard mathematical symbol in this context. It seems to be a typo or a placeholder for another symbol. Let's assume it's a placeholder for another symbol.
7. **^** symbol: This is a^\ symbol, which is not a standard mathematical symbol in this context. It seems to be a typo or a placeholder for another symbol. Let's assume it's a placeholder for another symbol.
8. **^** symbol: This is a^\ symbol, which is not a standard mathematical symbol in this context. It seems to be a typo or a placeholder for another symbol. Let's assume it's a placeholder for another symbol.
9. **^** symbol: This is a^\ symbol, which is not a standard mathematical symbol in this context. It seems to be a typo or a placeholder for another symbol. Let's assume it's a placeholder for another symbol.
10. **^** symbol: This is a^\ symbol, which is not a standard mathematical symbol in this context. It seems to


\section*{Lines 248–248 / inline(\$ \$) }
assistant
Let's break down the given equation step by step, following the order you've specified.

### Example
The equation given is:
\[
\mathbbm{1}_{ij}^{\text{ objective}}
\]

### Explanation
1. ** symbol**: This is the indicator function, which is often used in probability and statistics. It indicates whether a certain condition is met.
2. ** \( \mathbbm{1} \)**: This is the symbol for the indicator function. The double subscript \(ij\) indicates that the function is applied to the pair \((i, j)\).
3. ** \( \text{ objective} \)**: This is the label for the indicator function, which in this context is "objective".

So, the indicator function \(\mathbbm{1}_{ij}^{\text{ objective}}\) is 1 if the condition is met (i.e., if the pair \((i, j)\) is part of the "objective" set), and 0 otherwise.

### Conclusion
The core purpose of this equation is to indicate whether a certain pair \((i, j)\) is part of the "objective" set. If the pair is part of the "objective" set, the value of the indicator function is 1; otherwise, it is 0.

Therefore, the conclusion can be summarized as:
\[
\boxed{\text{The indicator function } \mathbbm{1}_{ij}^{\text{ objective}} \text{ indicates whether the pair } (i, j) \text{ is part of the "objective" set.}}
\]


\end{document}
