\\documentclass[11pt]{article}
\\usepackage[margin=1in]{geometry}
\\usepackage{amsmath, amssymb, amsfonts}
\\usepackage{hyperref}
\\usepackage{kotex}
\\setlength{\\parskip}{6pt}
\\setlength{\\parindent}{0pt}
\\title{LaTeX Equation Explanation Report (Middle-School Level+)}
\\author{Automatic Pipeline}
\\date{2025-09-11}
\\begin{document}
\\maketitle
\\tableofcontents
\\newpage

\\section*{Document Overview}
assistant
You are a clear, concise technical write-up that remaining a general technical audience.

\newpage

\section*{Lines 91–91 / inline(\$ \$) }
assistant
Let's break down the given equation step by step.

### Example
The equation is:
\[ 448 \times 448 \]

### Explanation
To explain this equation clearly, we can break it down into its components and their roles:

1. ** symbol**: This symbol represents multiplication. It tells us that we are multiplying the number 448 by itself.
2. ** symbol**: This symbol represents the number 448. It is the number we are multiplying.

So, the equation \( 448 \times 448 \) means we are multiplying the number 448 by itself. In other words, we are finding the square of 448.

### Conclusion
The core purpose of this equation is to calculate the square of the number 448. In the context of the paper, this could be useful for various purposes such as calculating the area of a square with a side length of 448 units, or it could be a part of a larger mathematical model where the square of 448 is used to represent some quantity or value.

So, the conclusion can be summarized as:
\[ \boxed{448^2} \]


\section*{Lines 118–118 / inline(\$ \$) }
assistant
Let's break down the given equation \( S \times S \) step by step.

### Example
\[ S \times S \]

### Explanation
The symbol \( S \) represents a set. The expression \( S \times S \) denotes the Cartesian product of the set \( S \) with itself. The Cartesian product of two sets \( A \) and \( B \) is the set of all ordered pairs \((a, b)\) where \( a \) is an element of \( A \) and \( b \) is an element of \( B \). In this case, since both sets are \( S \), the Cartesian product \( S \times S \) is the set of all ordered pairs \((s_1, s_2)\) where both \( s_1 \) and \( s_2 \) are elements of \( S \).

### Conclusion
The core purpose of the equation \( S \times S \) in the context of the paper is to represent the Cartesian product of a set with itself, which is a fundamental concept in set theory and can be used in various mathematical and computational applications.

\[
\boxed{S \times S}
\]


\section*{Lines 120–120 / inline(\$ \$) }
assistant
Let's break down the given equation step by step, following the order you've specified.

### Example
\[
\Pr(\textrm{ Object}) * \textrm{IOU}_{\textrm{推}}^{\textrm{实际}}
\]

### Explanation
1. **\(\Pr(\textrm{ Object})\)**: This represents the probability of the event "Object". In simpler terms, it's the likelihood that a certain event (the "Object") occurs.
2. **\(\textrm{IOU}_{\textrm{推}}^{\textrm{实际}}\)**: This represents the "initialization" or "initial condition" of the system, which is "推" (which translates to "推导" or "derives" in English). The exponent "实际" (which translates to "实际" or "actual") indicates that we are considering the actual or real condition of the system.

### Conclusion
The core purpose of this equation is to calculate the probability of an event (the "Object") given the initial condition of the system (the "推导" or "derives" condition). In other words, it's a way to determine the likelihood of a specific outcome based on the starting point of the system.

So, the final answer is:
\[
\boxed{\text{The probability of the "Object" given the "推导" condition.}}
\]


\section*{Lines 124–124 / inline(\$ \$) }
assistant
The equation \(\Pr(\textrm{Class}_i | \textrm{ Object})\) represents the probability that an object belongs to a certain class \(i\). Let's break it down step by step:

1. ** symbol**: This represents the probability function. The vertical bar \(|\) means "given that".

2. **\(\textrm{Class}_i\)**: This is the class we are considering. It is a specific category or label that the object can belong to.

3. **\(\textrm{Object}\)**: This is the object or data point we are considering. It is the item we are trying to classify or predict.

4. **\(\Pr\)**: This is the probability function. It tells us the likelihood of the object being in class \(i\) given the object itself.

So, in simpler terms, \(\Pr(\textrm{Class}_i | \textrm{Object})\) is the probability that the object belongs to class \(i\). This is a fundamental concept in machine learning and pattern recognition, where we often need to estimate the likelihood of an object belonging to a certain class based on the information we have about the object.

The final conclusion is that the equation represents the probability of an object belonging to a specific class, given the object itself. This is a crucial concept in many machine learning algorithms, particularly in Naive Bayes classification and other probabilistic models.

\[
\boxed{\Pr(\textrm{Class}_i | \textrm{Object})}
\]


\section*{Lines 127–130 / env [equation]}
assistant
Let's break down the given equation step by step:

### Equation:
\[
\Pr(\textrm{Class}_i | \textrm{Object}) * \Pr(\textrm{Object}) * \textrm{IOU}_{\textrm{推}}^{\textrm{真}} = \Pr(\textrm{Class}_i) * \textrm{IOU}_{\textrm{推}}^{\textrm{真}}
\]

### Explanation:
1. **Notation and meanings:**
   - \(\Pr(\textrm{Class}_i | \textrm{Object})\): The probability that an object belongs to class \(i\) given the object.
   - \(\Pr(\textrm{Object})\): The probability of the object.
   - \(\textrm{IOU}_{\textrm{推}}^{\textrm{真}}\): A term representing the information used for推 (推理) and its truth value.
   - \(\Pr(\textrm{Class}_i)\): The probability of classifying an object as class \(i\).

2. **Step-by-step breakdown:**
   - The left-hand side of the equation is the product of three terms:
     1. The probability that the object belongs to class \(i\) given the object.
     2. The probability of the object.
     3. The information used for推 and its truth value.
   - The right-hand side of the equation is the product of two terms:
     1. The probability of classifying the object as class \(i\).
     2. The information used for推 and its truth value.

3. **Conclusion:**
   - The equation states that the product of the probability of the object, the probability of the object being in class \(i\) given the object, and the information used for推 and its truth value is equal to the product of the probability of classifying the object as class \(i\) and the information used for推 and its truth value.

### Final Answer:
\[
\boxed{\Pr(\textrm{Class}_i | \textrm{ Object}) \cdot \Pr(\textrm{Object}) \cdot \textrm{IOU}_{\textrm{推}}^{\textrm{真}} = \Pr(\textrm{Class}_i


\section*{Lines 138–138 / inline(\$ \$) }
assistant
Let's break down the given equation \( S \times S \) step by step.

### Example
The equation \( S \times S \) is simply the product of \( S \) with itself. In other words, it means \( S \) multiplied by \( S \).

### Explanation
- \( S \) represents a variable or a term in the equation.
- The symbol \( \times \) indicates multiplication.
- Therefore, \( S \times S \) means \( S \) multiplied by \( S \), which is the same as \( S^2 \) (i.e., \( S \) squared).

### Conclusion
The core purpose of the equation \( S \times S \) is to represent the square of the variable \( S \). In the context of the paper, this could be useful for calculating the squared value of a particular quantity or variable.

So, the final answer is:

\[
\boxed{S^2}
\]


\section*{Lines 138–138 / inline(\$ \$) }
assistant
Let's break down the given equation step by step:

1. **Example:**
   \[
   S \times S \times (B*5 + C)
   \]

2. **Explanation:**
   - \( S \): This represents a variable or a constant.
   - \( S \times S \) means \( S \) is multiplied by itself, which is \( S^2 \).
   - \( B*5 \) means \( B \) is multiplied by 5.
   - \( C \): This is another constant.
   - \( B*5 + C \): This is the sum of \( B \) multiplied by 5 and \( C \).
   - The entire expression \( S^2 \times (B*5 + C) \) means \( S^2 \) is multiplied by the sum of \( B*5 + C \).

3. **Conclusion:**
   The core purpose of this equation is to calculate a value that is the product of \( S^2 \) and the sum of \( B*5 + C \). This could be used in various contexts, such as in machine learning algorithms where \( S \), \( B \), and \( C \) might represent different variables or parameters.

Final answer:
\[
\boxed{S^2 \times (B*5 + C)}
\]


\section*{Lines 142–142 / inline(\$ \$) }
assistant
Let's break down the given equation step by step.

### Example
The equation is:
\[ 7 \times 7 \times 30 \]

### Explanation
1. **7**: This is the first number in the multiplication.
2. **×**: This is the multiplication symbol.
3. **7**: This is the second number in the multiplication.
4. **×**: This is the multiplication symbol again.
5. **30**: This is the third number in the multiplication.

Now, let's perform the multiplication step by step:
\[ 7 \times 7 = 49 \]
\[ 49 \times 30 = 1470 \]

So, the value of the equation is:
\[ 1470 \]

### Conclusion
The core purpose of this equation is to calculate the product of the three numbers. In the context of the paper, it could be used to determine a loss, normalize data, calculate an error, or compute a probability or likelihood. However, without additional context, we can only conclude that it is a simple multiplication of the given numbers.

The final answer is:
\[
\boxed{1470}
\]


\section*{Lines 150–150 / inline(\$ \$) }
assistant
Let's break down the given equation step by step, following the order you've specified.

### Example
The equation is:
\[ 1 \times 1 \]

### Explanation
1. ** symbol**: This is the multiplication symbol. It indicates that we are multiplying the number 1 by the number 1.
2. ** symbol**: This is the number 1.
3. ** symbol**: This is the number 1.
4. ** symbol**: This is the multiplication operation, which tells us to multiply the two numbers that follow it.

So, the equation \(1 \times 1\) means we are multiplying the number 1 by the number 1, which results in 1.

### Conclusion
The core purpose of this equation in the context of the paper is to show that the loss composed of normalized coordinates error is minimized when the coordinates are close to each other. In other words, the equation demonstrates that the error between the coordinates is minimized when they are equal, which is a fundamental concept in many machine learning algorithms.

The final answer is:
\[
\boxed{1}
\]


\section*{Lines 150–150 / inline(\$ \$) }
assistant
Let's break down the given equation step by step, following the instructions provided.

### Example
The equation is:
\[ 224 \times 224 \]

### Explanation
To explain this equation clearly, we can break it down into its components and their meanings:

1. ** symbol**: This is the multiplication symbol. It indicates that we are multiplying the number 224 by itself.
2. ** number 224**: This is the number we are multiplying.

So, the equation \( 224 \times 224 \) means we are multiplying 224 by 224.

### Conclusion
The core purpose of this equation is to calculate the product of 224 and 224. In other words, it is a simple multiplication of the same number with itself, which is also known as squaring the number 224.

Therefore, the conclusion can be summarized as:
\[ \text{The result of multiplying 224 by 224 is 50176.} \]

Putting it all together, we have:

### Example
\[ 224 \times 224 \]

### Explanation
Multiplying 224 by 224.

### Conclusion
The result of multiplying 224 by 224 is 50176.

So, the final answer is:
\[ \boxed{50176} \]


\section*{Lines 156–156 / inline(\$ \$) }
assistant
Let's break down the given equation step by step.

### Example
The equation is:
\[ 1 \times 1 \]

### Explanation
The equation \(1 \times 1\) is a simple multiplication of two ones. Here's the breakdown:

1. The symbol \(\times\) represents multiplication.
2. The number 1 is being multiplied by itself.
3. The result of multiplying 1 by 1 is 1.

So, the equation \(1 \times 1\) simplifies to 1.

### Conclusion
The core purpose of this equation in the context of the paper is to show that multiplying 1 by 1 results in 1. This is a basic arithmetic operation that is often used as a building block in more complex mathematical expressions.

The final answer is:
\[
\boxed{1}
\]


\section*{Lines 156–156 / inline(\$ \$) }
assistant
Let's break down the given equation step by step.

### Example
The equation is:
\[ 3 \times 3 \]

### Explanation
1. ** symbol**: This is the multiplication symbol.
2. ** 3**: This is the first number in the multiplication.
3. ** ×**: This is the multiplication symbol.
4. ** 3**: This is the second number in the multiplication.

So, the equation \( 3 \times 3 \) means we are multiplying the number 3 by itself, which is the same as adding 3 to itself 3 times.

\[ 3 \times 3 = 3 + 3 + 3 = 9 \]

### Conclusion
The core purpose of this equation is to calculate the product of 3 and 3, which is 9. In the context of the paper, this equation could be used to calculate a loss, normalize data, or determine the error in a coordinate system, among other things. The specific purpose would depend on the context of the paper, but the equation itself is simply a multiplication of two numbers.

The final answer is:
\[
\boxed{9}
\]


\section*{Lines 160–160 / inline(\$ \$) }
assistant
Let's break down the given equation step by step.

### Example
The equation is:
\[ 7 \times 7 \times 30 \]

### Explanation
1. **7**: This is the first number in the multiplication.
2. **×**: This is the multiplication symbol.
3. **7**: This is the second number in the multiplication.
4. **×**: This is the multiplication symbol again.
5. **30**: This is the third number in the multiplication.

Now, let's calculate the value step by step:
\[ 7 \times 7 = 49 \]
\[ 49 \times 30 = 1470 \]

So, the value of the equation is:
\[ 1470 \]

### Conclusion
The core purpose of this equation is to calculate the product of the numbers 7, 7, and 30. In the context of the paper, this could be useful for various purposes such as calculating a loss, normalizing values, or determining an error in coordinates. The final answer is:
\[ \boxed{1470} \]


\section*{Lines 173–173 / inline(\$ \$) }
assistant
Let's break down the given equation step by step, following the order you've specified.

### Example
The equation is:
\[ 224 \times 224 \]

### Explanation
1. ** symbol**: This is the multiplication symbol.
2. ** 224**: This is the number being multiplied.
3. ** \times**: This is the multiplication symbol.
4. ** 224**: This is the second number being multiplied.
5. ** ****: This is the equals sign, indicating the result of the multiplication.

So, the equation in a clear and simple English explanation would be:
"Multiply 224 by 224 to get the result."

### Conclusion
The core purpose of this equation is to calculate the product of 224 and 224. In the context of the paper, this could be a part of a larger calculation or model, but without additional context, the main point is simply the result of the multiplication.

The final answer is:
\[ 224 \times 224 = 50176 \]

So, the conclusion can be summarized as:
\[ \boxed{50176} \]


\section*{Lines 173–173 / inline(\$ \$) }
assistant
Let's break down the given equation step by step.

### Example
The equation is \(448 \times 448\).

### Explanation
To explain this equation clearly, we can break it down into its components and their roles:

1. ** symbol**: This is the multiplication symbol. It indicates that we are multiplying the number 448 by itself.
2. ** symbol**: This is the number 448, which is the number we are multiplying.

So, the equation \(448 \times 448\) means we are multiplying 448 by 448.

### Conclusion
The core purpose of this equation is to calculate the product of 448 and 448. In other words, it is finding the square of 448.

Therefore, the conclusion is:

\[
\boxed{200704}
\]


\section*{Lines 179–185 / env [equation]}
assistant
Let's break down the given equation step by step.

### Example
The equation provided is:
\[
\phi(x) =
\begin{cases}
    x, & \text{if } x > 0\\
    0.1x, & \text{otherwise}
\end{cases}
\]

### Explanation
1. ** symbol**: This is the piecewise function notation, which means the function \(\phi(x)\) has different definitions depending on the value of \(x\).
2. ** if \(x > 0\)**: This is the first case. If \(x\) is greater than 0, then \(\phi(x) = x\). In other words, the function returns the value of \(x\) itself.
3. ** otherwise**: This is the second case. If \(x\) is not greater than 0 (i.e., \(x \leq 0\)), then \(\phi(x) = 0.1x\). In other words, the function returns 0.1 times the value of \(x\).

### Conclusion
The core purpose of this equation is to define a piecewise function that outputs \(x\) if \(x\) is positive and \(0.1x\) if \(x\) is non-positive. This function can be useful in various contexts, such as in machine learning or optimization problems where different outputs are needed based on the input value.

The final answer is:
\[
\boxed{\phi(x) =
\begin{cases}
    x, & \text{if } x > 0\\
    0.1x, & \text{otherwise}
\end{cases}}
\]


\section*{Lines 198–246 / env [multline]}
assistant
Let's break down the given equation step by step and provide an explanation for each part.

### Example
The given equation is:
\[
\lambda_\textbf{coord}
\sum_{i = 0}^{S^2}
    \sum_{j = 0}^{B}
     \mathlarger{\mathbbm{1}}_{ij}^{\text{ objective}}
            \left[
            \left(
                x_i - \hat{x}_i
            \right)^2 +
            \left(
                y_i - \hat{y}_i
            \right)^2
            \right]
\\
+ \lambda_\textbf{coord} 
\sum_{i = 0}^{S^2}
    \sum_{j = 0}^{B}
         \mathlarger{\mathbbm{1}}_{ij}^{\text{ objective}}
         \left[
        \left(
            \sqrt{w_i} - \sqrt{\hat{w}_i}
        \right)^2 +
        \left(
            \sqrt{h_i} - \sqrt{\hat{h}_i}
        \right)^2
        \right]
\\
+ \sum_{i = 0}^{S^2}
    \sum_{j = 0}^{B}
        \mathlarger{\mathbbm{1}}_{ij}^{\text{ objective}}
        \left(
            C_i - \hat{C}_i
        \right)^2
\\
+ \lambda_\textrm{no objective}
\sum_{i = 0}^{S^2}
    \sum_{j = 0}^{B}
    \mathlarger{\mathbbm{1}}_{ij}^{\text{ no objective}}
        \left(
            C_i - \hat{C}_i
        \right)^2
\\
+ \sum_{i = 0}^{S^2}
\mathlarger{\mathbbm{1}}_i^{\text{ objective}}
    \sum_{c \in \textrm{classes}}
        \left(
            p_i(c) - \hat{p}_i(c)
        \right)^2
\]

### Explanation
1. **Objective Function Terms:**
   - The first term is a weighted sum of squared differences between the observed and predicted coordinates \((x_i, y


\section*{Lines 248–248 / inline(\$ \$) }
assistant
Let's break down the given equation step by step, following the order you've specified.

### Example
The equation is:
\[
\mathbbm{1}_i^{\text{ objective}}
\]

### Explanation
1. ** symbol**: This is the indicator function, which is often used in probability and statistics. It indicates whether an event occurs or not. In this context, \(\mathbbm{1}_i\) is an indicator function that equals 1 if the event occurs and 0 otherwise.
2. ** i**: This is an index or a variable that represents a specific instance or element.
3. ** objective**: This is the objective function or the goal that the indicator function is associated with. It could be a loss function, a cost function, or any other function that we are trying to minimize or maximize.

### Conclusion
The core purpose of this equation is to represent an indicator function that is used to determine whether a specific event (or instance) \(i\) is part of the objective or not. If the event occurs, the indicator function equals 1; otherwise, it equals 0. This function is often used in optimization problems to constraint the solution space or to regularize the model.

So, the final answer is:
\[
\boxed{\text{The indicator function that indicates whether a specific instance } i \text{ is part of the objective or not.}}
\]


\section*{Lines 248–248 / inline(\$ \$) }
assistant
Let's break down the given equation step by step, following the order you've specified.

### Example
The equation is:
\[
\mathbbm{1}_{ij}^{\text{ objective}}
\]

### Explanation
1. ** symbol**: This is the indicator function, which is often used in probability and statistics. It indicates whether a certain condition is met.
2. ** \( \mathbbm{1} \)**: This is the symbol for the indicator function. The double subscript \(ij\) indicates that the function is applied to the pair \((i, j)\).
3. ** \( \text{ objective} \)**: This is the label for the indicator function, which in this context is "objective."

### Conclusion
The core purpose of this equation is to define an indicator function that is used to indicate whether a certain condition is met for a pair \((i, j)\) in the context of the paper. This function is labeled as "objective" and is used to measure the deviation or error in a certain context.

### Final Answer
\[
\boxed{\text{The indicator function } \mathbbm{1}_{ij}^{\text{ objective}} \text{ is used to indicate whether a certain condition is met for a pair } (i, j) \text{ in the context of the paper.}}
\]


\end{document}
