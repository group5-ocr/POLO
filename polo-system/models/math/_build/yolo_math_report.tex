\\documentclass[11pt]{article}
\\usepackage[margin=1in]{geometry}
\\usepackage{amsmath, amssymb, amsfonts}
\\usepackage{hyperref}
\\usepackage{kotex}
\\setlength{\\parskip}{6pt}
\\setlength{\\parindent}{0pt}
\\title{LaTeX Equation Explanation Report (Middle-School Level+)}
\\author{Automatic Pipeline}
\\date{2025-09-15}
\\begin{document}
\\maketitle
\\tableofcontents
\\newpage

\\section*{Document Overview}
assistant
\\end\{document\}

\newpage

\section*{Lines 91–91 / inline(\$ \$) }
assistant
Let's break down the given equation step by step.

### Example
The equation is \(448 \times 448\).

### Explanation
To explain this equation clearly, we can break it down into its components and their roles:

1. ** symbol**: This is the multiplication symbol. It indicates that we are multiplying the number 448 by itself.
2. ** symbol**: This is the number 448, which is the number we are multiplying.
3. ** symbol**: This is the result of the multiplication. So, \(448 \times 448 = 200704\).

### Conclusion
The core purpose of this equation is to calculate the square of the number 448. In other words, it finds the area of a square with a side length of 448 units.

So, the final answer is:

\[
\boxed{200704}
\]


\section*{Lines 118–118 / inline(\$ \$) }
assistant
Let's break down the given equation \( S \times S \) step by step.

### Example
The equation \( S \times S \) can be written exactly as LaTeX in a single block as:
\[ S \times S \]

### Explanation
- \( S \): This symbol represents a set or a space in the context of the paper. In this case, it is simply a variable representing a set or a space.
- \( \times \): This symbol represents the Cartesian product. The Cartesian product of a set \( S \) with itself, denoted as \( S \times S \), is the set of all ordered pairs \((s_1, s_2)\) where \( s_1 \) and \( s_2 \) are elements of \( S \).

So, \( S \times S \) is the set of all possible pairs of elements from the set \( S \).

### Conclusion
The core purpose of the equation \( S \times S \) in the context of the paper is to represent the Cartesian product of the set \( S \) with itself. This operation is often used in various mathematical and machine learning contexts, particularly when dealing with pairs of elements from a set.

Therefore, the conclusion can be summarized as:
\[ \text{The equation } S \times S \text{ represents the Cartesian product of the set } S \text{ with itself.} \]

So, the final answer is:
\[ \boxed{S \times S} \]


\section*{Lines 120–120 / inline(\$ \$) }
assistant
Let's break down the given equation step by step, following the order you've specified.

### Example
The equation is:
\[
\Pr(\textrm{ Object}) * \textrm{IOU}_{\textrm{推}}^{\textrm{ truth}}
\]

### Explanation
1. **\(\Pr(\textrm{ Object})\)**: This represents the probability of the event "Object". In simpler terms, it's the likelihood that the event "Object" occurs.
2. **\(\textrm{IOU}_{\textrm{推}}^{\textrm{ truth}}\)**: This term is a bit more complex. It seems to be a function or a specific notation that represents some kind of calculation or transformation involving "推" (推导) and "truth". Without more context, it's hard to be precise, but it likely represents some kind of information or data related to the "truth" or " truth" itself.

### Conclusion
The core purpose of this equation is to calculate the probability of an event "Object" and then multiply it by some information or data related to the "truth". In the context of the paper, this could be useful for tasks like machine learning, where you might need to calculate the probability of an event and then use that probability to make a decision or prediction.

### Final Answer
\[
\boxed{\text{The equation calculates the probability of an event "Object" and then multiplies it by some information or data related to the "truth".}}
\]


\section*{Lines 124–124 / inline(\$ \$) }
assistant
The equation \(\Pr(\textrm{Class}_i | \textrm{Object})\) represents the probability that an object belongs to a certain class \(i\), given the object. Here's a detailed explanation:

1. ** symbol**: This symbol represents the probability. The notation \(\Pr\) is used to denote probability.
2. ** symbol**: This symbol represents the class to which the object belongs. The index \(i\) indicates which specific class we are considering.
3. ** symbol**: This symbol represents the object. The context implies that we are considering the probability of the object being in class \(i\).

In simpler terms, this equation is calculating the likelihood that a given object belongs to a specific class. The probability is calculated based on the information provided in the context of the paper, which is not explicitly given here but is implied by the use of the probability notation.

**Conclusion**: The core purpose of this equation is to determine the probability that an object belongs to a specific class, given the information about the object and the classes it could belong to. This is a common task in machine learning and pattern recognition, where the goal is often to classify objects into categories based on the available data.

\[
\boxed{\Pr(\textrm{Class}_i | \textrm{Object})}
\]


\section*{Lines 127–130 / env [equation]}
assistant
Let's break down the given equation step by step:

### Example
\[
\Pr(\textrm{Class}_i | \textrm{ Object}) * \Pr(\textrm{Object}) * \textrm{IOU}_{\textrm{推}}^{\textrm{真}} = \Pr(\textrm{Class}_i) * \textrm{IOU}_{\textrm{推}}^{\textrm{真}}
\]

### Explanation
1. **\(\Pr(\textrm{Class}_i | \textrm{ Object})\)**: This represents the probability that the object belongs to class \(i\), given the object itself.
2. **\(\Pr(\textrm{Object})\)**: This represents the probability of the object itself.
3. **\(\textrm{IOU}_{\textrm{推}}^{\textrm{真}}\)**: This represents the information used for推断 (the information that is used to make the decision or prediction).

The left side of the equation is the product of the probability that the object belongs to class \(i\) given the object itself, the probability of the object, and the information used for推断. The right side of the equation is the product of the probability that the object belongs to class \(i\) and the information used for推断.

### Conclusion
The core purpose of this equation is to show that the probability of an object belonging to a certain class, given the object itself and the information used for推断, is equal to the probability of the object belonging to that class multiplied by the information used for推断. This equation is a fundamental part of the model's decision-making process, ensuring that the probability of a class assignment is consistent with the information available.

\[
\boxed{\Pr(\textrm{Class}_i | \textrm{ Object}) * \Pr(\textrm{Object}) * \textrm{IOU}_{\textrm{推}}^{\textrm{真}} = \Pr(\textrm{Class}_i) * \textrm{IOU}_{\textrm{推}}^{\textrm{真}}}
\]


\section*{Lines 138–138 / inline(\$ \$) }
assistant
Let's break down the given equation \( S \times S \) step by step.

### Example
\[ S \times S \]

### Explanation
The symbol \( S \) represents a set. The expression \( S \times S \) denotes the Cartesian product of the set \( S \) with itself. The Cartesian product of two sets \( A \) and \( B \) is the set of all ordered pairs \((a, b)\) where \( a \) is an element of \( A \) and \( b \) is an element of \( B \). In this case, since both sets are \( S \), the Cartesian product \( S \times S \) is the set of all ordered pairs \((s_1, s_2)\) where both \( s_1 \) and \( s_2 \) are elements of \( S \).

### Conclusion
The core purpose of the equation \( S \times S \) in the context of the paper is to represent the Cartesian product of a set with itself. This operation is often used in various mathematical and statistical contexts, particularly when dealing with sets of data or when performing operations on pairs of elements from the same set.

So, the final answer is:
\[ \boxed{S \times S} \]


\section*{Lines 138–138 / inline(\$ \$) }
assistant
Let's break down the given equation step by step:

1. **Example**: The equation is \( S \times S \times (B*5 + C) \).

2. **Explanation**:
   - \( S \): This represents a scalar value.
   - \( S \times S \): This means we are squaring the scalar value \( S \).
   - \( B \): This represents another scalar value.
   - \( 5 \): This is a constant multiplier for \( B \).
   - \( B*5 \): This means we are multiplying the scalar value \( B \) by 5.
   - \( C \): This represents another scalar value.
   - \( B*5 + C \): This means we are adding the scalar value \( C \) to the product of \( B \) and 5.
   - \( S \times S \times (B*5 + C) \): This means we are multiplying the squared value of \( S \) by the sum of \( B*5 \) and \( C \).

3. **Conclusion**: The core purpose of this equation is to calculate a weighted sum of squared values, where the weights are determined by the scalar values \( B \) and \( C \), and the scalar value \( S \) is squared before being multiplied by the sum of \( B*5 \) and \( C \).

The final answer is \(\boxed{S \times S \times (B*5 + C)}\).


\section*{Lines 142–142 / inline(\$ \$) }
assistant
Let's break down the given equation step by step.

### Example
The equation is: \(7 \times 7 \times 30\).

### Explanation
1. **7**: This is the first number in the multiplication.
2. **×**: This is the multiplication symbol.
3. **7**: This is the second number in the multiplication.
4. **×**: This is the multiplication symbol again.
5. **30**: This is the third number in the multiplication.

Now, let's perform the multiplication step by step:
1. First, multiply the first two numbers: \(7 \times 7 = 49\).
2. Then, multiply the result by the third number: \(49 \times 30 = 1470\).

So, the value of the equation \(7 \times 7 \times 30\) is \(1470\).

### Conclusion
The core purpose of this equation is to calculate the product of the numbers 7, 7, and 30. In the context of the paper, this could be useful for various purposes such as calculating a loss, normalizing values, or determining an error in a coordinate system. However, without additional context, the specific purpose is not clear. The conclusion would be that the equation is used to find the product of the three numbers.

The final answer is \(\boxed{1470}\).


\section*{Lines 150–150 / inline(\$ \$) }
assistant
Let's break down the given equation step by step.

### Example
The equation is:
\[ 1 \times 1 \]

### Explanation
The equation \( 1 \times 1 \) is a simple multiplication of two ones. Here's the breakdown:

1. The symbol \(\times\) represents multiplication.
2. The number 1 is being multiplied by itself.
3. The result of multiplying 1 by 1 is 1.

So, the equation simplifies to:
\[ 1 \]

### Conclusion
The core purpose of this equation in the context of the paper is to show that multiplying 1 by 1 results in 1. This is a basic arithmetic operation that is often used as a building block in more complex mathematical expressions.

The final answer is:
\[
\boxed{1}
\]


\section*{Lines 150–150 / inline(\$ \$) }
assistant
Let's break down the given equation step by step, following the instructions provided.

### Example
The equation is:
\[ 224 \times 224 \]

### Explanation
To explain this equation clearly, we can break it down into its components and their meaning:

1. ** symbol**: This is the multiplication symbol.
2. ** 224**: This is the number being multiplied by itself.

So, the equation \( 224 \times 224 \) means we are multiplying the number 224 by itself. In other words, we are calculating the square of 224.

### Conclusion
The core purpose of this equation in the context of the paper is to calculate the square of the number 224. This could be useful in various mathematical or computational contexts, such as calculating areas, probabilities, or error measures in machine learning models.

So, the conclusion can be summarized as:
\[ \text{The equation calculates the square of 224.} \]

### Final Answer
\[
\boxed{224 \times 224 = 50176}
\]


\section*{Lines 156–156 / inline(\$ \$) }
assistant
Let's break down the given equation step by step.

### Example
The equation is:
\[ 1 \times 1 \]

### Explanation
1. ** symbol**: This is the multiplication symbol. It indicates that we are multiplying the number 1 by the number 1.
2. ** symbol**: This is the number 1.
3. ** symbol**: This is the number 1.
4. ** symbol**: This is the multiplication operation, which is represented by the "times" symbol in this context.

So, the equation \(1 \times 1\) means we are multiplying the number 1 by the number 1, which results in 1.

### Conclusion
The core purpose of this equation in the context of the paper is to show that the loss composed of normalized coordinates error is minimized when the probability or likelihood is maximized. In simpler terms, it shows that the equation is used to find the maximum likelihood estimate, which is a common method in statistics to estimate the parameters of a model.

The final answer is:
\[
\boxed{1}
\]


\section*{Lines 156–156 / inline(\$ \$) }
assistant
Let's break down the given equation step by step.

### Example
The equation given is:
\[ 3 \times 3 \]

### Explanation
1. ** symbol**: This is the multiplication symbol.
2. ** 3**: This is the first number in the multiplication.
3. ** ×**: This is the multiplication symbol.
4. ** 3**: This is the second number in the multiplication.

So, the equation \( 3 \times 3 \) means we are multiplying the number 3 by itself.

### Conclusion
The result of multiplying 3 by 3 is 9. Therefore, the conclusion is:
\[ 3 \times 3 = 9 \]

So, the final answer is:
\[
\boxed{9}
\]


\section*{Lines 160–160 / inline(\$ \$) }
assistant
Let's break down the given equation step by step.

### Example
The equation is:
\[ 7 \times 7 \times 30 \]

### Explanation
1. **7**: This is the first number in the multiplication.
2. **×**: This is the multiplication symbol.
3. **7**: This is the second number in the multiplication.
4. **×**: This is the multiplication symbol again.
5. **30**: This is the third number in the multiplication.

Now, let's calculate the value step by step:
\[ 7 \times 7 = 49 \]
\[ 49 \times 30 = 1470 \]

So, the value of the equation is:
\[ 1470 \]

### Conclusion
The core purpose of this equation is to calculate the product of the three numbers. In the context of the paper, it could be used to determine a loss, normalize data, calculate an error, or compute a probability or likelihood. The specific purpose would depend on the context provided in the paper.

The final answer is:
\[
\boxed{1470}
\]


\section*{Lines 173–173 / inline(\$ \$) }
assistant
Let's break down the given equation step by step.

### Example
The equation is:
\[ 224 \times 224 \]

### Explanation
1. ** symbol**: This is the multiplication symbol.
2. ** 224**: This is the number being multiplied.
3. ** \times**: This is the multiplication symbol, which is the same as the "times" symbol.
4. ** 224**: This is the second number being multiplied.
5. ** =**: This is the equals sign, indicating the result of the multiplication.
6. ** 50176**: This is the result of multiplying 224 by 224.

So, the equation \( 224 \times 224 \) simplifies to \( 50176 \).

### Conclusion
The core purpose of this equation is to calculate the product of 224 and 224, which results in 50176. In the context of the paper, this equation could be used to determine the total area of a square with a side length of 224 units.

Final answer:
\[
\boxed{50176}
\]


\section*{Lines 173–173 / inline(\$ \$) }
assistant
Let's break down the given equation step by step.

### Example
The equation is:
\[ 448 \times 448 \]

### Explanation
To explain this equation clearly, we can break it down into its components:

1. ** symbol**: This is the multiplication symbol. It indicates that we are multiplying the number 448 by itself.
2. ** number**: The number being multiplied is 448.

So, the equation \( 448 \times 448 \) means we are multiplying 448 by 448.

### Conclusion
The core purpose of this equation is to calculate the product of 448 and 448. In other words, it is finding the square of 448.

So, the conclusion can be summarized as:
\[ \text{The equation calculates the square of 448.} \]

### Final Answer
\[
\boxed{448 \times 448 = 200704}
\]


\section*{Lines 179–185 / env [equation]}
assistant
Let's break down the given equation step by step.

### Example
The equation provided is:
\[
\phi(x) =
\begin{cases}
    x, & \text{if } x > 0\\
    0.1x, & \text{otherwise}
\end{cases}
\]

### Explanation
1. **Symbol \(\phi(x)\)**: This represents a function \(\phi\) that takes an input \(x\).
2. ** piecewise function**: The function \(\phi(x)\) is defined in two parts based on the value of \(x\).

#### Case 1: \(x > 0\)
- If \(x\) is greater than 0, the function \(\phi(x)\) returns \(x\) itself.
- This means for positive values of \(x\), \(\phi(x) = x\).

#### Case 2: \(x \leq 0\)
- If \(x\) is less than or equal to 0, the function \(\phi(x)\) returns \(0.1x\).
- This means for non-positive values of \(x\), \(\phi(x) = 0.1x\).

### Conclusion
The core purpose of this equation is to define a piecewise function that outputs \(x\) if \(x\) is positive and \(0.1x\) if \(x\) is non-positive. This function can be useful in various contexts, such as in machine learning or optimization problems where different behavior is needed depending on the sign of the input.

\[
\boxed{\phi(x) =
\begin{cases}
    x, & \text{if } x > 0\\
    0.1x, & \text{otherwise}
\end{cases}}
\]


\section*{Lines 181–184 / env [cases]}
assistant
Let's break down the given equation step by step.

### Example
The equation provided is:
\[ f(x) = \begin{cases} 
x & \text{if } x > 0 \\
0.1x & \text{otherwise}
\end{cases} \]

### Explanation
1. ** symbol**: This is the piecewise function notation. It means that the function \( f(x) \) is defined in two parts.
2. ** if \( x > 0 \)**: This is the condition for the first part of the function. If \( x \) is greater than 0, then the function returns \( x \).
3. ** 0.1x**: This is the expression for the second part of the function. If \( x \) is not greater than 0 (i.e., \( x \leq 0 \)), then the function returns \( 0.1x \).

### Conclusion
The core purpose of this equation is to define a function \( f(x) \) that returns \( x \) if \( x \) is positive and \( 0.1x \) if \( x \) is non-positive. This can be useful in various contexts, such as in machine learning where you might want to apply different transformations based on the value of a variable.

The final answer is:
\[ \boxed{f(x) = \begin{cases} 
x & \text{if } x > 0 \\
0.1x & \text{otherwise}
\end{cases}} \]


\section*{Lines 198–246 / env [multline]}
assistant
Let's break down the given equation step by step and explain each part clearly.

### Example
The given equation is:
\[
\lambda_\textbf{coord}
\sum_{i = 0}^{S^2}
    \sum_{j = 0}^{B}
     \mathlarger{\mathbbm{1}}_{ij}^{\text{ objective}}
            \left[
            \left(
                x_i - \hat{x}_i
            \right)^2 +
            \left(
                y_i - \hat{y}_i
            \right)^2
            \right]
\\
+ \lambda_\textbf{coord} 
\sum_{i = 0}^{S^2}
    \sum_{j = 0}^{B}
         \mathlarger{\mathbbm{1}}_{ij}^{\text{ objective}}
         \left[
        \left(
            \sqrt{w_i} - \sqrt{\hat{w}_i}
        \right)^2 +
        \left(
            \sqrt{h_i} - \sqrt{\hat{h}_i}
        \right)^2
        \right]
\\
+ \sum_{i = 0}^{S^2}
    \sum_{j = 0}^{B}
        \mathlarger{\mathbbm{1}}_{ij}^{\text{ objective}}
        \left(
            C_i - \hat{C}_i
        \right)^2
\\
+ \lambda_\textrm{no objective}
\sum_{i = 0}^{S^2}
    \sum_{j = 0}^{B}
    \mathlarger{\mathbbm{1}}_{ij}^{\text{ no objective}}
        \left(
            C_i - \hat{C}_i
        \right)^2
\\
+ \sum_{i = 0}^{S^2}
\mathlarger{\mathbbm{1}}_i^{\text{ objective}}
    \sum_{c \in \textrm{classes}}
        \left(
            p_i(c) - \hat{p}_i(c)
        \right)^2
\]

### Explanation
1. **Objective Function:**
   \[
   \lambda_\textbf{coord}
   \sum_{i = 0}^{S^2}


\section*{Lines 248–248 / inline(\$ \$) }
assistant
Let's break down the given equation step by step.

### Example
The equation is:
\[
\mathbbm{1}_i^{\text{ objective}}
\]

### Explanation
1. ** symbol**: This is the indicator function, which is 1 if the condition is true and 0 if the condition is false.
2. ** symbol**: The index \( i \) indicates the specific instance or element being considered.
3. ** symbol**: The term "objective" indicates the goal or the function we are evaluating.

So, the indicator function \(\mathbbm{1}_i^{\text{ objective}}\) is 1 if the \(i\)-th instance is the objective, and 0 otherwise.

### Conclusion
The core purpose of this equation is to indicate whether the \(i\)-th instance is the objective or not. If it is the objective, the value is 1; otherwise, it is 0.

\[
\boxed{\text{The indicator function } \mathbbm{1}_i^{\text{ objective}} \text{ is 1 if the } i\text{-th instance is the objective, and 0 otherwise.}}
\]


\section*{Lines 248–248 / inline(\$ \$) }
assistant
Let's break down the given equation step by step, following the order you've specified.

### Example
The equation is:
\[
\mathbbm{1}_{ij}^{\text{ objective}}
\]

### Explanation
1. ** symbol**: This is the indicator function, which is often used in probability and statistics. It is defined as:
   \[
   \mathbbm{1}_{ij}^{\text{ objective}} = 
   \begin{cases} 
   1 & \text{if } i = j \\
   0 & \text{if } i \neq j 
   \end{cases}
   \]
   Here, \( \mathbbm{1}_{ij}^{\text{ objective}} \) is an indicator function that equals 1 if \( i = j \) and 0 otherwise. This function is used to denote whether two indices are equal or not.

2. ** symbol**: The subscript \(ij\) indicates that the function is defined for pairs of indices \(i\) and \(j\).

3. ** symbol**: The term "objective" in the exponent indicates that this is an objective function, which is a function that we want to minimize or maximize in an optimization problem.

### Conclusion
The core purpose of this equation is to define an indicator function that helps in determining whether two indices are the same or different. This function is often used in optimization problems to check for equality constraints.

### Final Answer
\[
\boxed{\text{The indicator function } \mathbbm{1}_{ij}^{\text{ objective}} \text{ equals 1 if } i = j \text{ and 0 otherwise.}}
\]


\end{document}
