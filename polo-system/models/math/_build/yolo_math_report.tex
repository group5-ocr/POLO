\\documentclass[11pt]{article}
\\usepackage[margin=1in]{geometry}
\\usepackage{amsmath, amssymb, amsfonts}
\\usepackage{hyperref}
\\usepackage{kotex}
\\setlength{\\parskip}{6pt}
\\setlength{\\parindent}{0pt}
\\title{LaTeX Equation Explanation Report (Middle-School Level+)}
\\author{Automatic Pipeline}
\\date{2025-09-17}
\\begin{document}
\\maketitle
\\tableofcontents
\\newpage

\\section*{Document Overview}
assistant
\\end\{document\}

\newpage

\section*{Lines 120–120 / inline(\$ \$) }
assistant
### Example
$$
\Pr(\textrm{Object}) * \textrm{IOU}_{\textrm{pred}}^{\textrm{truth}}
$$
### Explanation
Let's break down the equation step by step:

1. \(\Pr(\textrm{ Object})\): This represents the probability that a given object is true. It is a measure of the likelihood that the object is correct or valid.

2. : This represents the initial condition or the initial probability of the object being true. It is the starting point for the calculation of the probability of the object being true.

The equation multiplies these two probabilities together. This means that the overall probability of the object being true is the product of the initial probability of the object being true and the probability of the object being true given the initial condition.

### Conclusion
The core purpose of this equation is to calculate the overall probability of an object being true by combining the initial probability of the object being true with the probability of the object being true given the initial condition. This can be useful in various applications where we need to determine the likelihood of an event or object being true based on initial conditions and subsequent probabilities.


\section*{Lines 124–124 / inline(\$ \$) }
assistant
### Example
$$
\Pr(\textrm{Class}_i | \textrm{Object})
$$
### Explanation
The equation \(\Pr(\textrm{Class}_i | \textrm{Object})\) is a conditional probability that gives the likelihood of an object being in a particular class \(i\) given that it is an object. The notation \(\Pr(\textrm{Class}_i | \textrm{Object})\) means the probability of the class \(i\) given that the object is an object. This is a fundamental concept in machine learning and pattern recognition, where we often need to estimate the probability of a class label given an input.

### Conclusion
The core purpose of this equation is to provide a measure of how likely it is that a given object belongs to a specific class. This is useful in many applications, such as image识别, natural language processing, and machine learning models. By calculating this probability, we can make predictions about the class of an object based on its features.

\(\boxed{\Pr(\textrm{Class}_i | \textrm{Object})}\)


\section*{Lines 127–130 / env [equation]}
assistant
### Example
$$
\scriptsize
\Pr(\textrm{Class}_i | \textrm{Object}) * \Pr(\textrm{Object}) * \textrm{IOU}_{\textrm{pred}}^{\textrm{truth}} = \Pr(\textrm{Class}_i)*\textrm{IOU}_{\textrm{pred}}^{\textrm{truth}}
$$
### Explanation
Let's break down the equation step by step:

1. **Probability of Class \(i\) given Object**: \(\Pr(\textrm{Class}_i | \textrm{Object})\) represents the probability that the object belongs to class \(i\).

2. **Probability of Object**: \(\Pr(\textrm{Object})\) represents the probability that the object is any object in the domain.

3. **Initial Probability of推 (hypothesis)**:  represents the initial probability that the object is the true class.

4. **Right-hand side of the equation**:  represents the product of the probability of the object being in class \(i\) and the initial probability that the object is the true class.

### Conclusion
The equation states that the product of the probability of the object being in class \(i\) given that it is an object, and the probability of the object being any object, is equal to the product of the probability of the object being in class \(i\) and the initial probability that the object is the true class. This equation is a statement of the law of total probability, which is a fundamental principle in probability theory.


\section*{Lines 179–185 / env [equation]}
assistant
### Example
$$
\phi(x) =
\begin{cases}
    x, & \text{if } x > 0\\
    0.1x, & \text{otherwise}
    \end{cases}
$$
### Explanation
The function \(\phi(x)\) is a piecewise function that outputs different values based on the input \(x\). Specifically:
- If \(x\) is greater than 0, \(\phi(x)\) returns \(x\).
- If \(x\) is less than or equal to 0, \(\phi(x)\) returns \(0.1x\).

### Conclusion
The function \(\phi(x)\) is a simple threshold function that scales the input \(x\) by a factor of 0.1 when \(x\) is non-positive and returns \(x\) when \(x\) is positive. This function can be useful in various applications where a linear relationship exists for positive values and a different relationship (in this case, a scaled linear relationship) for non-positive values.

\(\blacksquare\)


\section*{Lines 198–246 / env [multline]}
assistant
### Example
$$
\lambda_\textbf{coord}
\sum_{i = 0}^{S^2}
    \sum_{j = 0}^{B}
     \mathlarger{\mathbbm{1}}_{ij}^{\text{obj}}
            \left[
            \left(
                x_i - \hat{x}_i
            \right)^2 +
            \left(
                y_i - \hat{y}_i
            \right)^2
            \right]
\\
+ \lambda_\textbf{coord} 
\sum_{i = 0}^{S^2}
    \sum_{j = 0}^{B}
         \mathlarger{\mathbbm{1}}_{ij}^{\text{obj}}
         \left[
        \left(
            \sqrt{w_i} - \sqrt{\hat{w}_i}
        \right)^2 +
        \left(
            \sqrt{h_i} - \sqrt{\hat{h}_i}
        \right)^2
        \right]
\\
+ \sum_{i = 0}^{S^2}
    \sum_{j = 0}^{B}
        \mathlarger{\mathbbm{1}}_{ij}^{\text{obj}}
        \left(
            C_i - \hat{C}_i
        \right)^2
\\
+ \lambda_\textrm{noobj}
\sum_{i = 0}^{S^2}
    \sum_{j = 0}^{B}
    \mathlarger{\mathbbm{1}}_{ij}^{\text{noobj}}
        \left(
            C_i - \hat{C}_i
        \right)^2
\\
+ \sum_{i = 0}^{S^2}
\mathlarger{\mathbbm{1}}_i^{\text{obj}}
    \sum_{c \in \textrm{classes}}
        \left(
            p_i(c) - \hat{p}_i(c)
        \right)^2
$$
### Explanation
The equation is composed of several summations and indicator functions. Let's break it down step by step:

1. **First Summation:**
   \[
   \lambda_\textbf{coord} \sum_{i = 0}^{S^2} \sum_{j = 0}^{B} \mathlarger{\mathbbm{1}}_{ij}^{\text{Obj}} \left[ \left( x_i - \hat{x}_i \right)^2 + \left( y_i - \hat{y}_i \right)^2 \right]
   \]
   - \(\lambda_\textbf{coord}\) is a weight factor.
   - The double summation iterates over \(i\) from 0 to \(S^2\) and \(j\) from 0 to \(B\).
   - \(\mathlarger{\mathbbm{1}}_{ij}^{\text{ objective}}\) is an indicator function that is 1 if the pair \((i, j)\) is an objective coordinate and 0 otherwise.
   - The term inside the summation is the squared error between the true coordinate \((x_i, y_i)\) and the predicted coordinate \((\hat{x}_i, \hat{y}_i)\).

2. **Second Summation:**
   \[
   \lambda_\textbf{coord} \sum_{i = 0}^{S^2} \sum_{j = 0}^{B} \mathlarger{\mathbbm{1}}_{ij}^{\text{ objective}} \left[ \left( \sqrt{w_i} - \sqrt{\hat{w}_i} \right)^2 + \left( \sqrt{h_i} - \sqrt{\hat{h}_i} \right)^2 \right]
   \]
   - This term is similar to the first one but uses the square roots of the weights \(w_i\) and \(h_i\) instead of the coordinates.
   - The indicator function \(\mathlarger{\mathbbm{1}}_{ij}^{\text{ objective}}\) is the same


\end{document}
