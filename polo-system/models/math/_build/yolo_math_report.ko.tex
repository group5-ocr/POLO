\\documentclass[11pt]{article}
\\usepackage[margin=1in]{geometry}
\\usepackage{amsmath, amssymb, amsfonts}
\\usepackage{hyperref}
\\usepackage{kotex}
\\setlength{\\parskip}{6pt}
\\setlength{\\parindent}{0pt}
\\title{LaTeX Equation Explanation Report (Middle-School Level+)}
\\author{Automatic Pipeline}
\\date{2025-09-15}
\\begin{document}
\\maketitle
\\tableofcontents
\\newpage

\\section*{Document Overview}
assistant
You are a clear, concise technical write-up that remaining the paper's main focus on object detection.

\newpage

\section*{Lines 91–91 / inline(\$ \$) }
assistant
Let's break down the given equation step by step.

### Example
The equation is:
\[ 448 \times 448 \]

### Explanation
To explain this equation clearly, let's break it down into its components:

1. ** symbol**: This is the multiplication symbol. It indicates that we are multiplying the number 448 by itself.
2. ** symbol**: This is the number 448, which is being multiplied by itself.

So, the equation \( 448 \times 448 \) means we are multiplying 448 by 448.

### Conclusion
The core purpose of this equation is to calculate the square of the number 448. In other words, it finds the area of a square with a side length of 448 units.

Therefore, the conclusion can be summarized as:
\[ \text{The equation calculates the square of 448.} \]

### Final Answer
\[
\boxed{448 \times 448 = 200704}
\]

\section*{Lines 118–118 / inline(\$ \$) }
assistant
Let's break down the given equation \( S \times S \) step by step.

### Example
\[ S \times S \]

### Explanation
The symbol \( S \) represents a set. The expression \( S \times S \) denotes the Cartesian product of the set \( S \) with itself. The Cartesian product of two sets \( A \) and \( B \) is the set of all ordered pairs \((a, b)\) where \( a \) is an element of \( A \) and \( b \) is an element of \( B \). In this case, since both sets are \( S \), the Cartesian product \( S \times S \) is the set of all ordered pairs \((s_1, s_2)\) where \( s_1 \) and \( s_2 \) are elements of \( S \).

### Conclusion
The core purpose of the equation \( S \times S \) in the context of the paper is to represent the Cartesian product of a set with itself. This operation is often used in various mathematical and statistical contexts, particularly when dealing with sets of data or outcomes.

\[
\boxed{S \times S}
\]

\section*{Lines 120–120 / inline(\$ \$) }
assistant
Let's break down the given equation step by step, following the order you've specified.

### Example
The equation is:
\[
\Pr(\textrm{ Object}) * \textrm{IOU}_{\textrm{推}}^{\textrm{真}}
\]

### Explanation
1. **\(\Pr(\textrm{ Object})\)**: This represents the probability of the event "Object". In simpler terms, it's the likelihood that a certain event (the "Object") occurs.
2. **\(\textrm{IOU}_{\textrm{推}}^{\textrm{真}}\)**: This term is a bit more complex. It seems to be a function or a specific notation that we need to interpret. Given the context, it might represent a specific operation or value related to "推" (推) and "真" (真). However, without additional context, it's challenging to provide a precise interpretation. For now, let's assume it represents a value related to the operation "推" and "真".

### Conclusion
The core purpose of this equation is to calculate the probability of an event (the "Object") and then apply a specific operation ( related to "推" and "真") to this probability. The exact nature of this operation depends on the specific context in which this equation is used.

### Final Answer
\[
\boxed{\text{The probability of the "Object" multiplied by a value related to "推" and "真".}}
\]

\section*{Lines 124–124 / inline(\$ \$) }
assistant
Let's break down the given equation step by step, following the order provided:

### Example
The equation is:
\[
\Pr(\textrm{Class}_i | \textrm{ Object})
\]

### Explanation
This equation represents the probability that an object belongs to a specific class \(i\), given the object. Let's break it down further:

1. **\(\Pr\)**: This is the probability function.
2. **\(\textrm{Class}_i\)**: This represents the class we are interested in. The index \(i\) indicates which class we are considering.
3. **\(|\)**: This is the conditional probability symbol, meaning "given".
4. **\(\textrm{ Object}\)**: This represents the object we are considering.

So, the equation \(\Pr(\textrm{Class}_i | \textrm{ Object})\) means "the probability that the object belongs to class \(i\)".

### Conclusion
The core purpose of this equation is to calculate the probability that a given object belongs to a specific class. This is a common task in machine learning and pattern recognition, where we often need to classify objects into different categories based on their features.

\[
\boxed{\text{The probability that the object belongs to class } i}
\]

\section*{Lines 127–130 / env [equation]}
assistant
Let's break down the given equation step by step, following the order you've specified.

### Example
The equation is:
\[
\Pr(\textrm{Class}_i | \textrm{Object}) * \Pr(\textrm{Object}) * \textrm{IOU}_{\textrm{推}}^{\textrm{真}} = \Pr(\textrm{Class}_i) * \textrm{IOU}_{\textrm{推}}^{\textrm{真}}
\]

### Explanation
1. **\(\Pr(\textrm{Class}_i | \textrm{Object})\)**: This represents the probability that the object belongs to class \(i\), given that we know the object is in the set of objects.
2. **\(\Pr(\textrm{Object})\)**: This represents the probability that the object is in the set of objects.
3. **\(\textrm{IOU}_{\textrm{推}}^{\textrm{真}}\)**: This represents the information used for推断, and it is true.

The left side of the equation is the product of the probability that the object belongs to class \(i\) given the object is in the set of objects, the probability that the object is in the set of objects, and the information used for推断 that is true.

The right side of the equation is the product of the probability that the object belongs to class \(i\) and the information used for推断 that is true.

### Conclusion
The core purpose of this equation is to show that the probability of an object belonging to a certain class, given that the object is in the set of objects, and the probability of the object being in the set of objects, multiplied by the information used for推断 that is true, is equal to the probability of the object belonging to that class, multiplied by the information used for推断 that is true.

In simpler terms, it is saying that the probability of an object being in a certain class, given that it is in the set of objects, is proportional to the probability of the object being in the set of objects, and the information used for推断 that is true.

The final answer is:
\[
\boxed{\text{The probability of an object belonging to a certain class, given that it is in the set of objects, is proportional to the probability of the

\section*{Lines 138–138 / inline(\$ \$) }
assistant
Let's break down the given equation \( S \times S \) step by step.

### Example
The equation \( S \times S \) is simply the product of \( S \) with itself. In other words, it means \( S \) multiplied by \( S \).

### Explanation
- \( S \) is a variable, which represents some quantity or value.
- The symbol \( \times \) means multiplication.
- So, \( S \times S \) means \( S \) multiplied by \( S \), which is the same as \( S^2 \) (where \( S^2 \) is the square of \( S \)).

### Conclusion
The core purpose of the equation \( S \times S \) is to represent the square of the variable \( S \). In the context of the paper, this could be useful for calculating the area of a square with side length \( S \), or for other purposes where the square of a quantity is needed.

The final answer is:
\[
\boxed{S^2}
\]

\section*{Lines 138–138 / inline(\$ \$) }
assistant
Let's break down the given equation step by step:

1. **Example**: The equation is \( S \times S \times (B*5 + C) \).

2. **Explanation**:
   - \( S \): This represents a variable or a constant.
   - \( S \times S \): This means \( S \) is multiplied by itself, which is equivalent to \( S^2 \).
   - \( B*5 \): This means the variable \( B \) is multiplied by 5.
   - \( B*5 + C \): This means the result of \( B*5 \) is added to the constant \( C \).
   - The entire expression \( S^2 \times (B*5 + C) \) means that the result of \( S^2 \) is multiplied by the result of \( B*5 + C \).

3. **Conclusion**: The core purpose of this equation is to calculate a certain value based on the variables and constants involved. In the context of the paper, it could be related to a loss function, a normalization factor, or some other statistical or machine learning concept, but without additional context, we can only say that it involves squaring \( S \) and then multiplying it by the sum of \( B \) multiplied by 5 and \( C \).

So, the final answer is:

\[
\boxed{S^2 \times (5B + C)}
\]

\section*{Lines 142–142 / inline(\$ \$) }
assistant
Let's break down the given equation step by step.

### Example
\[ 7 \times 7 \times 30 \]

### Explanation
1. **7**: This is a constant number.
2. **×**: This is the multiplication symbol.
3. **7**: This is another constant number.
4. **×**: This is the multiplication symbol again.
5. **30**: This is another constant number.

So, the equation is simply the product of these three numbers:
\[ 7 \times 7 \times 30 \]

### Calculation
Let's calculate the value step by step:
\[ 7 \times 7 = 49 \]
\[ 49 \times 30 = 1470 \]

So, the value of the equation is:
\[ 1470 \]

### Conclusion
The core purpose of this equation is to calculate the product of the three constants 7, 7, and 30. In the context of the paper, this could be a loss composition, normalization, coordinate error, probability/log-likelihood, etc., depending on the specific application.

The final answer is:
\[ \boxed{1470} \]

\section*{Lines 150–150 / inline(\$ \$) }
assistant
Let's break down the given equation step by step, following the order you've specified.

### Example
The equation is:
\[ 1 \times 1 \]

### Explanation
1. ** symbol**: This is the multiplication symbol. It indicates that we are multiplying the number 1 by the number 1.
2. ** symbol**: This is the number 1.
3. ** symbol**: This is the number 1.
4. ** symbol**: This is the multiplication operation, which tells us to multiply the two numbers that follow it.

So, the equation \(1 \times 1\) means we are multiplying the number 1 by the number 1, which results in 1.

### Conclusion
The core purpose of this equation in the context of the paper is to show that the loss composed of normalized coordinates error is minimized when the probability or likelihood is maximized. In other words, the equation demonstrates that the loss is minimized when the model's predictions are as close as possible to the actual values, which is a fundamental principle in machine learning and statistical modeling.

The final answer is:
\[
\boxed{1}
\]

\section*{Lines 150–150 / inline(\$ \$) }
assistant
Let's break down the given equation step by step, following the instructions provided.

### Example
The equation is:
\[ 224 \times 224 \]

### Explanation
To explain this equation in a clear and concise way, we can break it down into its components and their meaning:

1. ** symbol**: This is the multiplication symbol.
2. ** 224**: This is the number being multiplied by itself.
3. ** \times**: This is the multiplication symbol, which is the same as the "times" symbol in English.
4. ** 224**: This is the number being multiplied by itself.
5. ** 224**: This is the number being multiplied by itself.

So, the equation \( 224 \times 224 \) means "224 multiplied by 224".

### Conclusion
The core purpose of this equation is to calculate the product of 224 and 224. In the context of the paper, this could be useful for various purposes such as calculating the area of a square with side length 224 units or for any other application where this specific multiplication is needed.

### Final Answer
\[
\boxed{50176}
\]

\section*{Lines 156–156 / inline(\$ \$) }
assistant
Let's break down the given equation step by step, following the order you've specified.

### Example
The equation is:
\[ 1 \times 1 \]

### Explanation
1. ** symbol**: This is the multiplication symbol. It indicates that we are multiplying the number 1 by the number 1.
2. ** symbol**: This is the number 1.
3. ** symbol**: This is the number 1.
4. ** symbol**: This is the multiplication operation, indicating that we are multiplying the two numbers that follow.

So, the equation \( 1 \times 1 \) means we are multiplying the number 1 by the number 1, which results in 1.

### Conclusion
The core purpose of this equation in the context of the paper is to show that the loss composed of normalized coordinates error is minimized when the coordinates are equal to 1. In other words, the equation demonstrates that the loss is at its minimum when the coordinates are 1.

The final answer is:
\[
\boxed{1}
\]

\section*{Lines 156–156 / inline(\$ \$) }
assistant
Let's break down the given equation step by step.

### Example
The equation given is:
\[ 3 \times 3 \]

### Explanation
1. ** symbol**: This is the multiplication symbol.
2. ** 3**: This is the first number in the multiplication.
3. ** ×**: This is the multiplication symbol.
4. ** 3**: This is the second number in the multiplication.

So, the equation \( 3 \times 3 \) means we are multiplying the number 3 by itself, which is the same as adding 3 to itself 3 times:
\[ 3 \times 3 = 3 + 3 + 3 = 9 \]

### Conclusion
The core purpose of this equation in the context of the paper is to calculate the product of two numbers, which in this case is 9.

### Final Answer
\[
\boxed{9}
\]

\section*{Lines 160–160 / inline(\$ \$) }
assistant
Let's break down the given equation step by step.

### Example
The equation is: \(7 \times 7 \times 30\).

### Explanation
1. **7**: This is the first number in the multiplication.
2. **×**: This is the multiplication symbol.
3. **7**: This is the second number in the multiplication.
4. **×**: This is the multiplication symbol again.
5. **30**: This is the third number in the multiplication.

So, the equation \(7 \times 7 \times 30\) means we need to multiply 7 by 7 first, and then multiply the result by 30.

Let's do the multiplication step by step:
1. \(7 \times 7 = 49\)
2. \(49 \times 30 = 1470\)

So, the value of the equation \(7 \times 7 \times 30\) is \(1470\).

### Conclusion
The core purpose of this equation in the context of the paper is to calculate the total number of elements in a specific structure or system. For example, if this equation represents the total number of elements in a 3D structure where each dimension is 7 units and the third dimension is 30 units, then the conclusion would be that the total number of elements is \(1470\).

The final answer is \(\boxed{1470}\).

\section*{Lines 173–173 / inline(\$ \$) }
assistant
Let's break down the given equation step by step, following the order you specified.

### Example
The equation is:
\[ 224 \times 224 \]

### Explanation
1. ** symbol**: This is the multiplication symbol.
2. ** 224**: This is the number being multiplied.
3. ** symbol**: This is the multiplication symbol again.
4. ** 224**: This is the second number being multiplied.

So, the equation \( 224 \times 224 \) means we are multiplying the number 224 by itself.

### Conclusion
The core purpose of this equation is to calculate the area of a square with side length 224 units. The area of a square is given by the formula \( \text{side} \times \text{side} \), which in this case is \( 224 \times 224 \).

Therefore, the conclusion is:
\[ \boxed{50176} \]

\section*{Lines 173–173 / inline(\$ \$) }
assistant
Let's break down the given equation step by step.

### Example
The equation is:
\[ 448 \times 448 \]

### Explanation
To explain this equation clearly, we can break it down into its components:

1. ** symbol**: This is the multiplication symbol.
2. ** 448**: This is the number being multiplied by itself.

So, the equation \( 448 \times 448 \) means we are multiplying the number 448 by itself. In other words, we are squaring the number 448.

### Conclusion
The core purpose of this equation is to calculate the square of the number 448. In the context of the paper, this could be useful for various mathematical operations or computations.

### Final Answer
\[
\boxed{448^2}
\]

\section*{Lines 179–185 / env [equation]}
assistant
Let's break down the given piecewise function step by step.

### Example
The given equation is:
\[
\phi(x) =
\begin{cases}
    x, & \text{if } x > 0\\
    0.1x, & \text{otherwise}
\end{cases}
\]

### Explanation
1. ** symbol**: This is the piecewise function notation. It means that the function \(\phi(x)\) has different definitions depending on the value of \(x\).
2. **if \(x > 0\)**: If the value of \(x\) is greater than 0, then \(\phi(x) = x\). This means that \(\phi(x)\) is equal to \(x\) itself.
3. **otherwise**: If the value of \(x\) is not greater than 0 (i.e., \(x \leq 0\)), then \(\phi(x) = 0.1x\). This means that \(\phi(x)\) is equal to 0.1 times \(x\).

### Conclusion
The core purpose of this equation is to define a function \(\phi(x)\) that behaves differently depending on the value of \(x\). Specifically, it returns \(x\) if \(x\) is positive, and \(0.1x\) if \(x\) is non-positive. This type of function is often used in machine learning and optimization to introduce non linearity or to weight certain parts of the input differently based on their value.

\[
\boxed{\phi(x) =
\begin{cases}
    x, & \text{if } x > 0\\
    0.1x, & \text{otherwise}
\end{cases}}
\]

\section*{Lines 181–184 / env [cases]}
assistant
Let's break down the given equation step by step:

### Example
The equation provided is:
\[ f(x) = \begin{cases} 
x & \text{if } x > 0 \\
0.1x & \text{otherwise}
\end{cases} \]

### Explanation
1. ** symbol**: This is the piecewise function notation, which means the function \( f(x) \) is defined in two parts.
2. ** if \( x > 0 \)**: This is the condition for the first part of the function. If \( x \) is greater than 0, then \( f(x) = x \).
3. ** 0.1x**: This is the expression for the second part of the function. If \( x \) is not greater than 0 (i.e., \( x \leq 0 \)), then \( f(x) = 0.1x \).

### Conclusion
The core purpose of this equation is to define a function \( f(x) \) that returns \( x \) if \( x \) is positive and \( 0.1x \) if \( x \) is non-positive. This can be useful in various contexts, such as in machine learning where you might want to apply different transformations based on the sign of a variable.

### Final Answer
\[
\boxed{f(x) = \begin{cases} 
x & \text{if } x > 0 \\
0.1x & \text{otherwise}
\end{cases}}
\]

\section*{Lines 198–246 / env [multline]}
assistant
Let's break down the given equation step by step, following the order of the output:

### Example
\[
\lambda_\textbf{coord}
\sum_{i = 0}^{S^2}
    \sum_{j = 0}^{B}
     \mathlarger{\mathbbm{1}}_{ij}^{\text{ objective}}
            \left[
            \left(
                x_i - \hat{x}_i
            \right)^2 +
            \left(
                y_i - \hat{y}_i
            \right)^2
            \right]
\\
+ \lambda_\textbf{coord} 
\sum_{i = 0}^{S^2}
    \sum_{j = 0}^{B}
         \mathlarger{\mathbbm{1}}_{ij}^{\text{ objective}}
         \left[
        \left(
            \sqrt{w_i} - \sqrt{\hat{w}_i}
        \right)^2 +
        \left(
            \sqrt{h_i} - \sqrt{\hat{h}_i}
        \right)^2
        \right]
\\
+ \sum_{i = 0}^{S^2}
    \sum_{j = 0}^{B}
        \mathlarger{\mathbbm{1}}_{ij}^{\text{ objective}}
        \left(
            C_i - \hat{C}_i
        \right)^2
\\
+ \lambda_\textrm{no objective}
\sum_{i = 0}^{S^2}
    \sum_{j = 0}^{B}
    \mathlarger{\mathbbm{1}}_{ij}^{\text{ no objective}}
        \left(
            C_i - \hat{C}_i
        \right)^2
\\
+ \sum_{i = 0}^{S^2}
\mathlarger{\mathbbm{1}}_i^{\text{ objective}}
    \sum_{c \in \textrm{classes}}
        \left(
            p_i(c) - \hat{p}_i(c)
        \right)^2
\]

### Explanation
1. **Objective Function**:
   - The equation starts with \(\lambda_\textbf{coord}\), which is a Lagrange multiplier.
   - It then defines the objective

\section*{Lines 248–248 / inline(\$ \$) }
assistant
Let's break down the given equation step by step.

### Example
The equation given is:
\[
\mathbbm{1}_i^{\text{ objective}}
\]

### Explanation
1. ** symbol**: This is the indicator function, which is often used in probability and statistics. The indicator function \(\mathbbm{1}_i\) is defined as:
   \[
   \mathbbm{1}_i = \begin{cases} 
   1 & \text{if } i \text{ satisfies some condition} \\
   0 & \text{otherwise}
   \end{cases}
   \]
   In this context, \(\mathbbm{1}_i^{\text{ objective}}\) indicates that the indicator function is applied to the condition "objective".

2. ** symbol**: The exponent \(\text{ objective}\) is the condition or the context in which the indicator function is applied. It could be a specific event or a set of criteria that the index \(i\) must satisfy.

### Conclusion
The core purpose of this equation is to indicate whether a certain condition (in this case, "objective") is satisfied for a given index \(i\). If the condition is met, the indicator function returns 1; otherwise, it returns 0. This is a common way to represent binary conditions in mathematical expressions.

So, the final answer is:
\[
\boxed{\text{The indicator function } \mathbbm{1}_i^{\text{ objective}} \text{ indicates whether the condition "objective" is satisfied for the index } i.}
\]

\section*{Lines 248–248 / inline(\$ \$) }
assistant
Let's break down the given equation step by step, following the instructions provided.

### Example
The equation given is:
\[
\mathbbm{1}_{ij}^{\text{ objective}}
\]

### Explanation
1. ** symbol**: This is an indicator function, often used in probability and statistics. It indicates whether a certain condition is met.
2. ** \( \mathbbm{1} \)**: This is the symbol for the indicator function. The double subscript \(ij\) indicates the specific element of the function, where \(i\) and \(j\) are indices.
3. ** \( \text{ objective} \)**: This is the label for the indicator function, which in this context is "objective."

### Conclusion
The core purpose of this equation is to define an indicator function that is used to label a specific element in a matrix or table. In the context of the paper, this function is used to identify a particular outcome or event labeled as "objective."

### Final Answer
\[
\boxed{\text{The indicator function } \mathbbm{1}_{ij}^{\text{ objective}} \text{ is used to label a specific element in a matrix or table, indicating an "objective" outcome.}}
\]

\end{document}
