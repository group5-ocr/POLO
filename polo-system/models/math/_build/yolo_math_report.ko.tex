\\documentclass[11pt]{article}
\\usepackage[margin=1in]{geometry}
\\usepackage{amsmath, amssymb, amsfonts}
\\usepackage{hyperref}
\\usepackage{kotex}
\\setlength{\\parskip}{6pt}
\\setlength{\\parindent}{0pt}
\\title{LaTeX 수식 설명 보고서 (Middle-School Level+)}
\\author{자동 파이프라인}
\\date{2025-09-15}
\\begin{document}
\\maketitle
\\tableofcontents
\\newpage

\\section*{문서 개요}
도우미
\\end\{문서\}

\새 페이지

\section*{91–91행 / 인라인(\$ \$) }
assistant
주어진 방정식을 단계별로 분석해 보겠습니다.

### 예제
방정식은 \(448 \times 448\)입니다.

### 설명
이 방정식을 명확하게 설명하기 위해 방정식을 구성 요소와 그 역할로 나누어 보겠습니다.

1. ** 기호**: 곱셈 기호입니다. 448을 그 자체로 곱한다는 것을 나타냅니다.
2. ** 기호**: 곱하는 숫자인 448을 나타냅니다.
3. ** 기호**: 곱셈의 결과입니다. 따라서 \(448 \times 448 = 200704\)가 됩니다.

### 결론
이 방정식의 핵심 목적은 숫자 448의 제곱을 계산하는 것입니다. 즉, 한 변의 길이가 448인 정사각형의 넓이를 구하는 것입니다.

따라서 최종 답변은 다음과 같습니다.

⟦수학0⟧

\section*{118–118번째 줄 / 인라인(\$ \$) }
assistant
주어진 방정식 \( S \times S \)을 단계별로 분석해 보겠습니다.

### 예시
\( S \times S \) 방정식은 LaTeX에서 한 블록으로 정확하게 다음과 같이 작성할 수 있습니다.
\[ S \times S \]

### 설명
- \( S \): 이 기호는 본 논문의 맥락에서 집합 또는 공간을 나타냅니다. 이 경우, 단순히 집합 또는 공간을 나타내는 변수입니다.
- \( \times \): 이 기호는 데카르트 곱을 나타냅니다. 집합 \( S \)와 자기 자신에 대한 데카르트 곱은 \( S \times S \)로 표시되며, \( s_1 \)와 \( s_2 \)가 \( S \)의 원소인 모든 순서쌍 \((s_1, s_2)\)의 집합입니다.

따라서 \( S \times S \)는 \( S \) 집합의 모든 가능한 원소 쌍의 집합입니다.

### 결론
본 논문에서 방정식 \( S \times S \)의 핵심 목적은 집합 \( S \)의 자기 자신과의 데카르트 곱을 나타내는 것입니다. 이 연산은 다양한 수학 및 머신 러닝 맥락에서 자주 사용되며, 특히 집합의 원소 쌍을 다룰 때 유용합니다.

따라서 결론은 다음과 같이 요약할 수 있습니다.
\[ \text{The equation } S \times S \text{ represents the Cartesian product of the set } S \text{ with itself.} \]

따라서 최종 답은 다음과 같습니다.
\[ \boxed{S \times S} \]

\section*{120–120번째 줄 / 인라인(\$ \$) }
assistant
지정한 순서대로 주어진 방정식을 단계별로 분석해 보겠습니다.

### 예시
방정식은 다음과 같습니다.
\[
\Pr(\textrm{ Object}) * \textrm{IOU}_{\textrm{推}}^{\textrm{ truth}}
\]

### 설명
1. **\(\Pr(\textrm{ Object})\)**: "Object" 사건의 확률을 나타냅니다. 간단히 말해서 "Object" 사건이 발생할 가능성을 나타냅니다.
2. **\(\textrm{IOU}_{\textrm{推}}^{\textrm{ truth}}\)**: 이 용어는 좀 더 복잡합니다. "推" (推导)와 "truth"를 포함하는 일종의 계산이나 변환을 나타내는 함수 또는 특정 표기법처럼 보입니다. 더 자세한 맥락이 없으면 정확하게 설명하기 어렵지만, "truth" 또는 "truth" 자체와 관련된 일종의 정보나 데이터를 나타낼 가능성이 높습니다.

### 결론
이 방정식의 핵심 목적은 "객체" 사건의 확률을 계산한 다음, "진실"과 관련된 정보나 데이터를 곱하는 것입니다. 본 논문의 맥락에서, 이는 머신 러닝과 같이 사건의 확률을 계산한 다음, 그 확률을 사용하여 의사 결정이나 예측을 해야 하는 작업에 유용할 수 있습니다.

### 최종 답변
\[
\boxed{\text{The equation calculates the probability of an event "Object" and then multiplies it by some information or data related to the "truth".}}
\]

\section*{124–124행 / 인라인(\$ \$)}
보조
\(\Pr(\textrm{Class}_i | \textrm{Object})\) 방정식은 주어진 객체가 특정 클래스 \(i\)에 속할 확률을 나타냅니다. 자세한 설명은 다음과 같습니다.

1. ** 기호**: 이 기호는 확률을 나타냅니다. \(\Pr\) 표기법은 확률을 나타내는 데 사용됩니다.
2. ** 기호**: 이 기호는 객체가 속하는 클래스를 나타냅니다. 인덱스 \(i\)는 고려 중인 특정 클래스를 나타냅니다.
3. ** 기호**: 이 기호는 객체를 나타냅니다. 문맥상 객체가 \(i\) 클래스에 속할 확률을 고려하고 있음을 의미합니다.

간단히 말해서, 이 방정식은 주어진 객체가 특정 클래스에 속할 가능성을 계산하는 것입니다. 확률은 본 논문의 맥락에서 제공된 정보를 기반으로 계산되는데, 본 논문에서는 명시적으로 제시되지 않았지만 확률 표기법을 사용함으로써 함축적으로 표현됩니다.

**결론**: 이 방정식의 핵심 목적은 객체에 대한 정보와 객체가 속할 수 있는 클래스 정보를 바탕으로 객체가 특정 클래스에 속할 확률을 구하는 것입니다. 이는 머신 러닝과 패턴 인식에서 흔히 사용되는 작업으로, 이용 가능한 데이터를 기반으로 객체를 범주별로 분류하는 것이 목표인 경우가 많습니다.

⟦수학0⟧

\section*{127~130행 / env [방정식]}
assistant
주어진 방정식을 단계별로 분석해 보겠습니다.

### 예시
\[
\Pr(\textrm{Class}_i | \textrm{ Object}) * \Pr(\textrm{Object}) * \textrm{IOU}_{\textrm{推}}^{\textrm{真}} = \Pr(\textrm{Class}_i) * \textrm{IOU}_{\textrm{推}}^{\textrm{真}}
\]

### 설명
1. **\(\Pr(\textrm{Class}_i | \textrm{ Object})\)**: 객체 자체가 주어졌을 때, 해당 객체가 \(i\) 클래스에 속할 확률을 나타냅니다.
2. **\(\Pr(\textrm{Object})\)**: 객체 자체의 확률을 나타냅니다.
3. **\(\textrm{IOU}_{\textrm{推}}^{\textrm{真}}\)**: (the information that is used to make the decision or prediction)를 구하는 데 사용되는 정보를 나타냅니다.

방정식의 좌변은 객체 자체, 객체의 확률, 그리고 추론에 사용된 정보가 주어졌을 때, 객체가 클래스 \(i\)에 속할 확률의 곱입니다. 방정식의 우변은 객체가 클래스 \(i\)에 속할 확률과 추론에 사용된 정보의 곱입니다.

### 결론
이 방정식의 핵심 목적은 객체 자체와 추론에 사용된 정보를 고려할 때, 객체가 특정 클래스에 속할 확률이 해당 클래스에 속할 확률에 추론에 사용된 정보를 곱한 값과 같음을 보여주는 것입니다. 이 방정식은 모델의 의사결정 과정에서 핵심적인 부분으로, 클래스 할당 확률이 사용 가능한 정보와 일치하도록 보장합니다.

⟦수학0⟧

\section*{138–138행 / 인라인(\$ \$) }
assistant
주어진 방정식 \( S \times S \)을 단계별로 분석해 보겠습니다.

### 예시
\[ S \times S \]

### 설명
기호 \( S \)는 집합을 나타냅니다. 표현식 \( S \times S \)는 집합 \( S \)와 자기 자신 사이의 데카르트 곱을 나타냅니다. 두 집합 \( A \)와 \( B \)의 데카르트 곱은 모든 순서쌍 \((a, b)\)의 집합입니다. 여기서 \( a \)는 \( A \)의 원소이고 \( b \)는 \( B \)의 원소입니다. 이 경우 두 집합 모두 \( S \)이므로, 데카르트 곱 \( S \times S \)은 모든 순서쌍 \((s_1, s_2)\)의 집합입니다. 여기서 \( s_1 \)과 \( s_2 \)는 모두 \( S \)의 원소입니다.

### 결론
본 논문에서 \( S \times S \) 방정식의 핵심 목적은 한 집합과 자기 자신 사이의 데카르트 곱을 나타내는 것입니다. 이 연산은 다양한 수학적 및 통계적 맥락에서 자주 사용되며, 특히 데이터 집합을 다루거나 동일 집합의 원소 쌍에 대한 연산을 수행할 때 자주 사용됩니다.

따라서 최종 답은 다음과 같습니다.
\[ \boxed{S \times S} \]

\section*{138–138행 / 인라인(\$ \$) }
assistant
주어진 방정식을 단계별로 분석해 보겠습니다.

1. **예**: 방정식은 \( S \times S \times (B*5 + C) \)입니다.

2. **설명**:
- \( S \): 이것은 스칼라 값을 나타냅니다.
- \( S \times S \): 이것은 스칼라 값 \( S \)를 제곱한다는 것을 의미합니다.
- \( B \): 이것은 또 다른 스칼라 값을 나타냅니다.
- \( 5 \): 이것은 \( B \)의 상수 승수입니다.
- \( B*5 \): 이것은 스칼라 값 \( B \)에 5를 곱한다는 것을 의미합니다.
- \( C \): 이것은 또 다른 스칼라 값을 나타냅니다.
- \( B*5 + C \): 스칼라 값 \( C \)을 \( B \)과 5의 곱에 더한다는 의미입니다.
- \( S \times S \times (B*5 + C) \): \( S \)의 제곱 값에 \( B*5 \)와 \( C \)의 합을 곱한다는 의미입니다.

3. **결론**: 이 방정식의 핵심 목적은 제곱 값의 가중 합을 계산하는 것입니다. 여기서 가중치는 스칼라 값 \( B \)와 \( C \)에 의해 결정되고, 스칼라 값 \( S \)는 \( B*5 \)와 \( C \)의 합에 곱해지기 전에 제곱됩니다.

최종 답은 \(\boxed{S \times S \times (B*5 + C)}\)입니다.

\section*{142–142행 / 인라인(\$ \$) }
assistant
주어진 방정식을 단계별로 분석해 보겠습니다.

### 예시
방정식은 다음과 같습니다: \(7 \times 7 \times 30\).

### 설명
1. **7**: 곱셈의 첫 번째 숫자입니다.
2. **×**: 곱셈 기호입니다.
3. **7**: 곱셈의 두 번째 숫자입니다.
4. **×**: 역시 곱셈 기호입니다.
5. **30**: 곱셈의 세 번째 숫자입니다.

이제 곱셈을 단계별로 수행해 보겠습니다.
1. 먼저 처음 두 숫자를 곱합니다: \(7 \times 7 = 49\).
2. 그런 다음 결과에 세 번째 숫자를 곱합니다: \(49 \times 30 = 1470\).

따라서 방정식 \(7 \times 7 \times 30\)의 값은 \(1470\)입니다.

### 결론
이 방정식의 핵심 목적은 7, 7, 30의 곱을 계산하는 것입니다. 본 논문의 맥락에서 이는 손실 계산, 값 정규화, 좌표계 오차 확인 등 다양한 목적에 유용할 수 있습니다. 그러나 추가적인 맥락 없이는 구체적인 목적은 명확하지 않습니다. 결론적으로 이 방정식은 세 숫자의 곱을 구하는 데 사용된다고 할 수 있습니다.

최종 답은 \(\boxed{1470}\)입니다.

\section*{150–150번째 줄 / 인라인(\$ \$) }
assistant
주어진 방정식을 단계별로 분석해 보겠습니다.

### 예시
방정식은 다음과 같습니다.
\[ 1 \times 1 \]

### 설명
\( 1 \times 1 \) 방정식은 두 개의 1을 곱한 간단한 방정식입니다. 분석 결과는 다음과 같습니다.

1. 기호 \(\times\)는 곱셈을 나타냅니다.
2. 숫자 1은 자기 자신을 곱합니다.
3. 1에 1을 곱한 결과는 1입니다.

따라서 방정식은 다음과 같이 단순화됩니다.
\[ 1 \]

### 결론
본 논문에서 이 방정식의 핵심 목적은 1에 1을 곱하면 1이 된다는 것을 보여주는 것입니다. 이는 더 복잡한 수학 표현식의 구성 요소로 자주 사용되는 기본적인 산술 연산입니다.

최종 답은 다음과 같습니다.
\[
\boxed{1}
\]

\section*{150–150번째 줄 / 인라인(\$ \$) }
assistant
제공된 지침에 따라 주어진 방정식을 단계별로 분석해 보겠습니다.

### 예시
방정식은 다음과 같습니다.
\[ 224 \times 224 \]

### 설명
이 방정식을 명확하게 설명하기 위해 방정식의 구성 요소와 그 의미를 다음과 같이 나누어 볼 수 있습니다.

1. ** 기호**: 곱셈 기호입니다.
2. ** 224**: 자기 자신을 곱하는 숫자입니다.

따라서 \( 224 \times 224 \) 방정식은 224를 그 자체로 곱한다는 것을 의미합니다. 다시 말해, 224의 제곱을 계산하는 것입니다.

### 결론
본 논문에서 이 방정식의 핵심 목적은 224의 제곱을 계산하는 것입니다. 이는 머신 러닝 모델에서 면적, 확률 또는 오차 측정값을 계산하는 등 다양한 수학적 또는 계산적 맥락에서 유용할 수 있습니다.

따라서 결론은 다음과 같이 요약할 수 있습니다.
\[ \text{The equation calculates the square of 224.} \]

### 최종 답변
\[
\boxed{224 \times 224 = 50176}
\]

\section*{156–156행 / 인라인(\$ \$) }
assistant
주어진 방정식을 단계별로 분석해 보겠습니다.

### 예시
방정식은 다음과 같습니다.
\[ 1 \times 1 \]

### 설명
1. ** 기호**: 곱셈 기호입니다. 숫자 1에 숫자 1을 곱한다는 것을 나타냅니다.
2. ** 기호**: 숫자 1입니다.
3. ** 기호**: 숫자 1입니다.
4. ** 기호**: 곱셈 연산이며, 이 문맥에서는 "곱하기" 기호로 표현됩니다.

따라서 \(1 \times 1\) 방정식은 숫자 1에 숫자 1을 곱해서 1이 나온다는 것을 의미합니다.

### 결론
본 논문에서 이 방정식의 핵심 목적은 확률 또는 우도가 최대화될 때 정규화된 좌표 오차로 구성된 손실이 최소화됨을 보여주는 것입니다. 간단히 말해서, 이 방정식은 통계학에서 모델의 매개변수를 추정하는 일반적인 방법인 최대 우도 추정치를 구하는 데 사용됨을 보여줍니다.

최종 답은 다음과 같습니다.
\[
\boxed{1}
\]

\section*{156–156행 / 인라인(\$ \$) }
assistant
주어진 방정식을 단계별로 분석해 보겠습니다.

### 예시
주어진 방정식은 다음과 같습니다.
\[ 3 \times 3 \]

### 설명
1. ** 기호**: 곱셈 기호입니다.
2. ** 3**: 곱셈의 첫 번째 숫자입니다.
3. ** ×**: 곱셈 기호입니다.
4. ** 3**: 곱셈의 두 번째 숫자입니다.

따라서 \( 3 \times 3 \) 방정식은 숫자 3을 그 자체로 곱한다는 것을 의미합니다.

### 결론
3에 3을 곱한 결과는 9입니다. 따라서 결론은 다음과 같습니다.
\[ 3 \times 3 = 9 \]

따라서 최종 답은 다음과 같습니다.
\[
\boxed{9}
\]

\section*{160–160번째 줄 / 인라인(\$ \$) }
assistant
주어진 방정식을 단계별로 분석해 보겠습니다.

### 예시
방정식은 다음과 같습니다.
\[ 7 \times 7 \times 30 \]

### 설명
1. **7**: 곱셈의 첫 번째 숫자입니다.
2. **×**: 곱셈 기호입니다.
3. **7**: 곱셈의 두 번째 숫자입니다.
4. **×**: 역시 곱셈 기호입니다.
5. **30**: 곱셈의 세 번째 숫자입니다.

이제 값을 단계별로 계산해 보겠습니다.
\[ 7 \times 7 = 49 \]
\[ 49 \times 30 = 1470 \]

따라서 방정식의 값은 다음과 같습니다.
\[ 1470 \]

### 결론
이 방정식의 핵심 목적은 세 숫자의 곱을 계산하는 것입니다. 본 논문의 맥락에서 이 방정식은 손실 결정, 데이터 정규화, 오류 계산, 또는 확률 또는 우도 계산에 사용될 수 있습니다. 구체적인 목적은 논문에서 제공하는 맥락에 따라 달라집니다.

최종 답은 다음과 같습니다.
\[
\boxed{1470}
\]

\section*{173–173행 / 인라인(\$ \$) }
assistant
주어진 방정식을 단계별로 분석해 보겠습니다.

### 예시
방정식은 다음과 같습니다.
\[ 224 \times 224 \]

### 설명
1. ** 기호**: 곱셈 기호입니다.
2. ** 224**: 곱해지는 숫자입니다.
3. ** \times**: 곱셈 기호이며, "배" 기호와 같습니다.
4. ** 224**: 두 번째로 곱해지는 숫자입니다.
5. ** =**: 등호로, 곱셈 결과를 나타냅니다.
6. ** 50176**: 224에 224를 곱한 결과입니다.

따라서 방정식 \( 224 \times 224 \)는 \( 50176 \)로 단순화됩니다.

### 결론
이 방정식의 핵심 목적은 224와 224의 곱인 50176을 계산하는 것입니다. 이 논문의 맥락에서, 이 방정식은 한 변의 길이가 224인 정사각형의 총 면적을 구하는 데 사용될 수 있습니다.

최종 정답:
\[
\boxed{50176}
\]

\section*{173–173행 / 인라인(\$ \$) }
assistant
주어진 방정식을 단계별로 분석해 보겠습니다.

### 예시
방정식은 다음과 같습니다.
\[ 448 \times 448 \]

### 설명
이 방정식을 명확하게 설명하기 위해 방정식을 다음과 같이 구성 요소로 나눌 수 있습니다.

1. ** 기호**: 곱셈 기호입니다. 448이라는 숫자를 그 자체로 곱한다는 것을 나타냅니다.
2. ** 숫자**: 곱해지는 숫자는 448입니다.

따라서 \( 448 \times 448 \) 방정식은 448에 448을 곱한다는 것을 의미합니다.

### 결론
이 방정식의 핵심 목적은 448과 448의 곱을 계산하는 것입니다. 즉, 448의 제곱을 구하는 것입니다.

따라서 결론은 다음과 같이 요약할 수 있습니다.
\[ \text{The equation calculates the square of 448.} \]

### 최종 답변
\[
\boxed{448 \times 448 = 200704}
\]

\section*{179~185행 / env [방정식]}
assistant
주어진 방정식을 단계별로 분석해 보겠습니다.

### 예시
제공된 방정식은 다음과 같습니다.
\[
\phi(x) =
\begin{cases}
    x, & \text{if } x > 0\\
    0.1x, & \text{otherwise}
\end{cases}
\]

### 설명
1. **기호 \(\phi(x)\)**: 입력 \(x\)를 받는 함수 \(\phi\)를 나타냅니다.
2. **조각 함수**: 함수 \(\phi(x)\)는 \(x\)의 값을 기반으로 두 부분으로 정의됩니다.

#### 사례 1: \(x > 0\)
- \(x\)이 0보다 크면, 함수 \(\phi(x)\)는 \(x\) 자신을 반환합니다.
- 이는 \(x\), \(\phi(x) = x\)의 양수 값에 대해 다음을 의미합니다.

#### 사례 2: \(x \leq 0\)
- \(x\)가 0보다 작거나 같으면 \(\phi(x)\) 함수는 \(0.1x\)를 반환합니다.
- 이는 \(x\), \(\phi(x) = 0.1x\)가 양수가 아닌 값인 경우를 의미합니다.

### 결론
이 방정식의 핵심 목적은 \(x\)가 양수이면 \(x\)를 출력하고, \(x\)가 양수가 아니면 \(0.1x\)를 출력하는 구간 함수를 정의하는 것입니다. 이 함수는 입력의 부호에 따라 다른 동작이 필요한 머신 러닝이나 최적화 문제 등 다양한 상황에서 유용하게 사용될 수 있습니다.

⟦수학0⟧

\section*{181~184행 / env [cases]}
assistant
주어진 방정식을 단계별로 분석해 보겠습니다.

### 예시
제공된 방정식은 다음과 같습니다.
\[ f(x) = \begin{cases} 
x & \text{if } x > 0 \\
0.1x & \text{otherwise}
\end{cases} \]

### 설명
1. ** 기호**: 조각 함수 표기법입니다. 즉, 함수 \( f(x) \)가 두 부분으로 정의되어 있음을 의미합니다.
2. ** \( x > 0 \)인 경우**: 함수의 첫 번째 부분에 대한 조건입니다. \( x \)가 0보다 크면 함수는 \( x \)를 반환합니다.
3. ** 0.1x**: 함수의 두 번째 부분에 대한 표현식입니다. \( x \)가 0 (i.e., \( x \leq 0 \))보다 크지 않으면 함수는 \( 0.1x \)를 반환합니다.

### 결론
이 방정식의 핵심 목적은 \( x \)가 양수이면 \( x \)을 반환하고, \( x \)가 양수가 아니면 \( 0.1x \)을 반환하는 함수 \( f(x) \)를 정의하는 것입니다. 이는 변수 값에 따라 다양한 변환을 적용해야 하는 머신 러닝과 같은 다양한 상황에서 유용할 수 있습니다.

최종 답은 다음과 같습니다.
\[ \boxed{f(x) = \begin{cases} 
x & \text{if } x > 0 \\
0.1x & \text{otherwise}
\end{cases}} \]

\section*{198~246행 / env [multline]}
assistant
주어진 방정식을 단계별로 분석하고 각 부분을 명확하게 설명해 보겠습니다.

### 예제
주어진 방정식은 다음과 같습니다.
\[
\lambda_\textbf{coord}
\sum_{i = 0}^{S^2}
    \sum_{j = 0}^{B}
     \mathlarger{\mathbbm{1}}_{ij}^{\text{ objective}}
            \left[
            \left(
                x_i - \hat{x}_i
            \right)^2 +
            \left(
                y_i - \hat{y}_i
            \right)^2
            \right]
\\
+ \lambda_\textbf{coord} 
\sum_{i = 0}^{S^2}
    \sum_{j = 0}^{B}
         \mathlarger{\mathbbm{1}}_{ij}^{\text{ objective}}
         \left[
        \left(
            \sqrt{w_i} - \sqrt{\hat{w}_i}
        \right)^2 +
        \left(
            \sqrt{h_i} - \sqrt{\hat{h}_i}
        \right)^2
        \right]
\\
+ \sum_{i = 0}^{S^2}
    \sum_{j = 0}^{B}
        \mathlarger{\mathbbm{1}}_{ij}^{\text{ objective}}
        \left(
            C_i - \hat{C}_i
        \right)^2
\\
+ \lambda_\textrm{no objective}
\sum_{i = 0}^{S^2}
    \sum_{j = 0}^{B}
    \mathlarger{\mathbbm{1}}_{ij}^{\text{ no objective}}
        \left(
            C_i - \hat{C}_i
        \right)^2
\\
+ \sum_{i = 0}^{S^2}
\mathlarger{\mathbbm{1}}_i^{\text{ objective}}
    \sum_{c \in \textrm{classes}}
        \left(
            p_i(c) - \hat{p}_i(c)
        \right)^2
\]

### 설명
1. **목적 함수:**
\[
\lambda_\textbf{coord}
\sum_{i = 0}^{S^2}

\section*{248~248행 / 인라인(\$ \$) }
assistant
주어진 방정식을 단계별로 분석해 보겠습니다.

### 예시
방정식은 다음과 같습니다.
\[
\mathbbm{1}_i^{\text{ objective}}
\]

### 설명
1. ** 기호**: 이는 지시 함수로, 조건이 참이면 1, 거짓이면 0입니다.
2. ** 기호**: 인덱스 \( i \)는 고려 중인 특정 인스턴스 또는 요소를 나타냅니다.
3. ** 기호**: "목적"이라는 용어는 평가 중인 목표 또는 함수를 나타냅니다.

따라서 지표 함수 \(\mathbbm{1}_i^{\text{ objective}}\)는 \(i\)번째 인스턴스가 목적 함수이면 1이고, 그렇지 않으면 0입니다.

### 결론
이 방정식의 핵심 목적은 \(i\)번째 인스턴스가 목적 함수인지 아닌지를 나타내는 것입니다. 목적 함수이면 값은 1이고, 그렇지 않으면 0입니다.

⟦수학0⟧

\section*{248~248행 / 인라인(\$ \$) }
assistant
지정하신 순서대로 주어진 방정식을 단계별로 분석해 보겠습니다.

### 예시
방정식은 다음과 같습니다.
\[
\mathbbm{1}_{ij}^{\text{ objective}}
\]

### 설명
1. ** 기호**: 이는 확률 및 통계에서 자주 사용되는 지표 함수입니다. 다음과 같이 정의됩니다.
\[
   \mathbbm{1}_{ij}^{\text{ objective}} = 
   \begin{cases} 
   1 & \text{if } i = j \\
   0 & \text{if } i \neq j 
   \end{cases}
   \]
여기서 \( \mathbbm{1}_{ij}^{\text{ objective}} \)은 \( i = j \)이면 1이고, 그렇지 않으면 0인 지표 함수입니다. 이 함수는 두 지수가 같은지 여부를 나타내는 데 사용됩니다.

2. ** 기호 **: 아래 첨자 \(ij\)는 함수가 인덱스 쌍 \(i\)와 \(j\)에 대해 정의되었음을 나타냅니다.

3. **기호**: 지수의 "목적"이라는 용어는 이것이 목적 함수임을 나타냅니다. 목적 함수란 최적화 문제에서 최소화하거나 최대화하려는 함수입니다.

### 결론
이 방정식의 핵심 목적은 두 지수가 같은지 다른지 판단하는 데 도움이 되는 지표 함수를 정의하는 것입니다. 이 함수는 최적화 문제에서 등식 제약 조건을 확인하는 데 자주 사용됩니다.

### 최종 답변
\[
\boxed{\text{The indicator function } \mathbbm{1}_{ij}^{\text{ objective}} \text{ equals 1 if } i = j \text{ and 0 otherwise.}}
\]

\end{문서}
