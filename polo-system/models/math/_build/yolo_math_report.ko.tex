\\documentclass[11pt]{article}
\\usepackage[margin=1in]{geometry}
\\usepackage{amsmath, amssymb, amsfonts}
\\usepackage{hyperref}
\\usepackage{kotex}
\\setlength{\\parskip}{6pt}
\\setlength{\\parindent}{0pt}
\\title{LaTeX 수식 설명 보고서 (Middle-School Level+)}
\\author{자동 파이프라인}
\\date{2025-09-18}
\\begin{document}
\\maketitle
\\tableofcontents
\\newpage

\\section*{문서 개요}
보조원
원래 문서와 동일하게 유지되는 명확하고 간결한 기술 문서를 작성해 주세요.

\새 페이지

\section*{102~106행 / env [align]}
보조자
### 설명

주어진 방정식은 다음과 같습니다.

⟦수학0⟧
⟦수학1⟧

방정식의 각 부분을 단계별로 분석해 보겠습니다.

1. **합 표기법 이해하기:**
- 기호 \(\sum\)는 수열의 합을 나타냅니다.
- 색인 \(n\)는 1에서 시작하여 \(N\)까지 올라갑니다.
- \(\mathbb{E}(X_n^2)\)는 확률 변수 \(X_n\)의 제곱의 기댓값을 나타냅니다.
- \(\overline{X}_n\)는 확률 변수 \(X_n\)의 표본 평균을 나타냅니다.
- \(\overline{P}\)는 확률 변수 \(P\)의 표본 비율을 나타냅니다.

2. **부등식 해석:**
- 첫 번째 부등식은 \(N\) 확률 변수 \(X_n\)의 제곱에 대한 기댓값의 합이 \(N\)에 상수 \(P\)의 제곱에 대한 기댓값을 곱한 값보다 작거나 같다는 것을 나타냅니다.
- 두 번째 부등식은 \(N\) 확률 변수 \(\overline{X}_n\)의 표본 평균 제곱에 대한 기댓값의 합이 \(N\)에 표본 비율 \(\overline{P}\)의 제곱에 대한 기댓값을 곱한 값보다 작거나 같다는 것을 나타냅니다.

3. **결론:**
- 두 부등식 모두 본질적으로 확률 변수 (or their sample means)의 제곱에 대한 기댓값의 합이 상수 또는 표본 비율의 제곱에 대한 기댓값의 선형 함수로 제한됨을 나타냅니다.

### 최종 답변

⟦수학0⟧

조수
최종 답은 \(\boxed{\sum_{n=1}^N\mathbb{E}(X_n^2) \leq N\cdot P, \quad \sum_{n=1}^N\mathbb{E}(\overline{X}_n^2) \leq N\cdot \overline{P}}\)입니다.

\end{문서}
