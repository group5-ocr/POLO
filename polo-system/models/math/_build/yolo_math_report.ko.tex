\\documentclass[11pt]{article}
\\usepackage[margin=1in]{geometry}
\\usepackage{amsmath, amssymb, amsfonts}
\\usepackage{hyperref}
\\usepackage{kotex}
\\setlength{\\parskip}{6pt}
\\setlength{\\parindent}{0pt}
\\title{LaTeX 수식 설명 보고서 (Middle-School Level+)}
\\author{자동 파이프라인}
\\date{2025-09-19}
\\begin{document}
\\maketitle
\\tableofcontents
\\newpage

\\section*{문서 개요}
보조원
일반 기술 독자를 대상으로 명확하고 간결한 기술 문서를 작성합니다.

\새 페이지

\section*{120–120행 / 인라인(\$ \$) }
보조
제공된 LaTeX 논문의 맥락 내에서 주어진 방정식을 설명하려면 방정식과 제공된 맥락을 신중하게 분석해야 합니다. 방정식은 다음과 같습니다.

⟦수학0⟧

이 방정식은 예측된 객체의 신뢰도 점수를 나타냅니다. 단계별로 분석해 보겠습니다.

1. **\(\Pr(\textrm{Object})\)**: 이는 해당 객체가 격자 셀에 존재할 확률을 나타냅니다. 객체가 존재하지 않으면 이 확률은 0입니다.

2. **\(\textrm{IOU}_{\textrm{Pred}}^{\textrm{truth}}\)**: 예측된 상자와 실제 값 사이의 Intersection over Union (IOU) 점수를 나타냅니다. IOU 점수는 예측된 상자가 실제 값 상자와 얼마나 잘 일치하는지를 나타내는 척도입니다. 예측된 상자와 실제 값 상자의 교점 면적을 예측된 상자와 실제 값 상자의 합집합 면적으로 나누어 계산합니다.

방정식 전체는 이 두 항의 곱입니다. 즉, 예측된 객체의 신뢰도 점수는 해당 객체가 그리드 셀에 존재할 확률에 예측된 상자와 실제 상자 사이의 유사도 점수 (IOU)를 곱한 값입니다.

### 설명

이 방정식은 예측된 객체의 신뢰도 점수를 계산합니다. 신뢰도 점수는 해당 객체가 그리드 셀에 존재할 확률과 예측된 상자가 실제 상자와 얼마나 일치하는지를 나타내는 척도입니다. 객체가 존재하지 않으면 확률 항은 0이므로 신뢰도 점수도 0입니다. 객체가 존재하면 확률 항은 0과 1 사이의 값이 되고, 신뢰도 점수는 예측된 상자와 실제 상자 사이의 유사도 점수입니다.

### 변수 목록

- **\(\Pr(\textrm{Object})\)**: 객체가 그리드 셀에 존재할 확률입니다.
- **\(\textrm{IOU}_{\textrm{Pred}}^{\textrm{truth}}\)**: 예측된 상자와 실제 상자 사이의 합집합 점수에 대한 교집합입니다.

### 결론

이 방정식은 예측된 객체의 신뢰도 점수를 계산합니다. 객체가 존재하지 않으면 신뢰도 점수는 0입니다. 객체가 존재하는 경우, 신뢰도 점수는 객체가 존재할 확률과 예측된 상자와 실제 상자 사이의 유사도 점수의 곱입니다.

\[
\boxed{\Pr(\textrm{Object}) * \textrm{IOU}_{\textrm{Pred}}^{\textrm

\section*{124–124행 / 인라인(\$ \$) }
보조
논문의 맥락에서 주어진 방정식을 설명하려면 기호와 그 의미를 이해해야 합니다. 방정식은 다음과 같습니다.

⟦수학0⟧

이는 객체가 객체일 때, 객체가 \(i\) 클래스에 속할 확률을 나타냅니다. 논문의 맥락은 다음과 같은 정보를 제공합니다.

1. 이 논문은 이미지에서의 객체 검출에 관한 것입니다.
2. 이 방정식은 객체가 객체일 때, 특정 클래스 \(i\) 객체가 존재할 확률을 나타냅니다.

논문의 맥락은 기호 \(\Pr(\textrm{Class}_i | \textrm{Object})\) 자체를 정의하지는 않지만, 방정식을 이해하는 데 필요한 맥락을 제공합니다. 논문에서는 이 모델이 각 객체에 대해 각 클래스 \(i\)의 확률을 예측한다고 명시합니다. \(\Pr(\textrm{Class}_i | \textrm{Object})\) 표기법은 해당 객체가 객체일 때 확률이 조건부임을 나타냅니다.

이 방정식을 설명하자면 다음과 같습니다.

방정식 \(\Pr(\textrm{Class}_i | \textrm{Object})\)는 객체가 객체일 때 클래스 \(i\)에 속할 확률을 나타냅니다. 이는 논문의 맥락에서 제공되는 조건부 확률입니다. 논문에서는 이 모델이 각 객체에 대해 각 클래스 \(i\)에 속할 확률을 예측하며, 이 확률은 객체가 객체라는 조건 하에 존재한다고 명시합니다.

논문의 맥락은 기호 \(\Pr(\textrm{Class}_i | \textrm{Object})\) 자체를 정의하지는 않지만, 방정식을 이해하는 데 필요한 맥락을 제공합니다. 논문에서는 이 모델이 각 객체에 대해 각 클래스 \(i\)의 확률을 예측한다고 명시합니다. \(\Pr(\textrm{Class}_i | \textrm{Object})\) 표기법은 해당 객체가 객체일 때 확률이 조건부임을 나타냅니다.

따라서 이 방정식에 대한 설명은 다음과 같습니다.

⟦수학0⟧

보조
논문의 맥락은 방정식을 이해하는 데 필요한 맥락을 제공합니다. 논문은 모델이 각 객체에 대해 각 클래스 \(i\)의 확률을 예측한다고 명시합니다. \(\Pr(\textrm{Class}_i | \textrm{Object})\) 표기법은 해당 객체가 객체일 때 확률이 조건임을 나타냅니다.

따라서 이 방정식에 대한 설명은 다음과 같습니다.

⟦수학0⟧

\section*{127–130행 / env [방정식]}
보조
논문의 맥락 내에서 주어진 방정식을 설명하려면 방정식을 분해하고 그 구성 요소를 이해해야 합니다. 방정식은 다음과 같습니다.

단계별 설명은 다음과 같습니다.

1. **\(\Pr(\textrm{Class}_i | \textrm{Object})\)**: 이미지에 객체가 존재한다는 가정 하에, 객체가 특정 클래스 \(i\)에 속할 확률을 나타냅니다. 이는 조건부 확률입니다.

2. **\(\Pr(\textrm{Object})\)**: 이미지에 물체가 존재할 확률을 나타냅니다. 이는 이미지에 물체가 존재할 확률입니다.

3. ****: 예측된 상자의 합집합 (IOU) 점수에 대한 교집합을 나타냅니다. IOU 점수는 예측된 상자가 객체에 얼마나 잘 적합한지를 나타내는 척도입니다. 예측된 상자와 객체의 교집합 면적을 예측된 상자와 객체의 합집합 면적으로 나누어 계산합니다.

4. **\(\Pr(\textrm{Class}_i)\)**: 이는 상자가 \(i\) 클래스를 포함할 확률을 나타냅니다. 이는 상자가 특정 클래스를 포함할 사건의 확률입니다.

5. ****: 이는 예측된 상자의 합집합 점수에 대한 교집합과 동일합니다.

이 방정식은 이미지에 객체가 있을 때 객체가 특정 클래스에 속할 확률, 이미지에 객체가 있을 확률, 예측된 상자의 IOU 점수를 곱한 값은 상자가 해당 클래스를 포함할 확률에 예측된 상자의 IOU 점수를 곱한 값과 같다는 것을 나타냅니다.

이 방정식은 이미지의 각 상자에 대한 신뢰도 점수를 계산하는 데 사용됩니다. 신뢰도 점수는 상자가 해당 클래스를 포함할 확률과 예측된 상자의 IOU 점수를 합한 값입니다. 이를 통해 모델은 클래스의 확률과 상자가 해당 객체에 적합한지 여부를 모두 고려하여 상자의 점수를 매길 수 있습니다.

결론적으로, 이 방정식은 이미지의 각 상자에 대한 신뢰도 점수를 계산하는 데 사용되며, 상자가 해당 클래스를 포함할 확률과 예측된 상자의 IOU 점수를 고려합니다. 이를 통해 모델은 클래스의 확률과 상자가 객체에 적합한지 여부를 모두 고려하여 상자의 점수를 매길 수 있습니다.

최종 답변은 다음과 같습니다.

\section*{179–185행 / env [방정식]}
보조
주어진 방정식을 설명하려면 맥락과 사용된 기호를 이해해야 합니다. 제공된 방정식은 누설 정류 선형 유닛 (leaky ReLU) 함수를 정의하는 조각 함수입니다. 누설 ReLU 함수는 피드포워드 및 피드포워드 모델의 학습을 돕기 위해 음수 입력에 대해 작은 기울기를 갖는 정류 선형 유닛 ( ReLU )의 한 유형입니다.

누수 ReLU 함수는 다음과 같이 정의됩니다.
\[
\phi(x) =
\begin{cases}
    x, & \text{if } x > 0\\
    0.1x, & \text{otherwise}
    \end{cases}
\]

즉, 양수 입력 \( x \)에 대해 함수는 입력 값 \( x \) 자체를 반환합니다. 양수가 아닌 입력 \( x \) (i.e., \( x \leq 0 \))에 대해 함수는 작은 양수 값인 \( 0.1x \)를 반환합니다.

기능을 단계별로 나누어 보겠습니다.

1. **사례 1: \( x > 0 \)**
- 입력 \( x \)가 양수이면 함수는 입력 값 \( x \)를 반환합니다.
- 예를 들어, \( x = 5 \)이면 \( \phi(5) = 5 \)가 반환됩니다.

2. **사례 2: \( x \leq 0 \)**
- 입력 \( x \)가 양수가 아닌 (i.e., zero or negative)이면 함수는 \( 0.1x \)를 반환합니다.
- 예를 들어, \( x = -5 \)이면 \( \phi(-5) = 0.1 \times (-5) = -0.5 \)를 반환합니다.

누수 함수 ReLU는 머신 러닝 모델에서 모델에 비선형성을 도입하는 데 사용됩니다. 음수 입력에 대한 작은 기울기는 기울기가 너무 작아지는 것을 방지하여 딥 피드 포워드 모델의 학습에 도움이 될 수 있습니다.

본 논문의 맥락에서, 누설 함수 ReLU는 피드포워드 모델에서 이전 계층의 출력을 처리하는 데 사용됩니다. 이 함수는 모델에 비선형성을 도입하는 데 도움이 되며, 이는 출력이 클래스에 대한 확률 분포여야 하는 이미지 분류와 같은 작업에 중요합니다.

누수 ReLU 함수는 다음과 같이 정의됩니다.
\[
\phi(x) =
\begin{cases}
    x, & \text{if } x > 0\\
    0.1x, & \text{otherwise}
    \end{cases}
\]

이 함수는 피드포워드 모델에서 이전 계층의 출력을 처리하는 데 사용됩니다. 이 함수는 모델에 비선형성을 도입하는 데 도움이 되며, 이는 출력이 다음 값에 대한 확률 분포여야 하는 이미지 분류와 같은 작업에 중요합니다.

assistant
Leaky ReLAP 함수는 피드포워드 모델에서 이전 계층의 출력을 처리하는 데 사용됩니다. 이 함수는 모델에 비선형성을 도입하는 데 도움이 되며, 이는 출력이 클래스에 대한 확률 분포여야 하는 이미지 분류와 같은 작업에 중요합니다. Leaky ReLAP 함수는 다음과 같이 정의됩니다.
\[
\phi(x) =
\begin{cases}
    x, & \text{if } x > 0\\
    0.1x, & \text{otherwise}
    \end{cases}
\]

이 함수는 피드 포워드 모델에서 이전 계층의 출력을 처리하는 데 사용됩니다. 이 함수는 모델에 비선형성을 도입하는 데 도움이 되며, 이는 출력을 정확하게 표현해야 하는 이미지 분류와 같은 작업에 중요합니다.

\section*{181–184행 / env [cases]}
assistant
주어진 방정식을 설명하려면 맥락과 사용된 기호를 이해해야 합니다. 방정식을 단계별로 분석해 보겠습니다.

주어진 방정식은 다음과 같습니다.
\[
\phi(x) = \begin{cases} 
x, & \text{if } x > 0\\
0.1x, & \text{otherwise}
\end{cases}
\]

이는 \(x\)가 0보다 크면 \(x\)를 출력하고, 그렇지 않으면 \(0.1x\)를 출력하는 구간 함수입니다. 다시 말해, \(x\)의 양수 값에 대해 기울기가 1이고, \(x\)의 양수가 아닌 값에 대해 기울기가 0.1인 선형 함수입니다.

### 설명

1. **로컬 컨텍스트**: 논문 컨텍스트에서는 경계 상자 \(x\) 및 \(y\) 좌표가 특정 그리드 셀 위치에서 오프셋되어 있으며 0과 1 사이로 제한된다고 언급합니다. 경계 상자는 모델 출력이 예상 범위 내에 있는지 확인하는 데 사용됩니다.

2. **함수의 구성 요소**:
- \(x > 0\)이면 함수는 \(x\)을 출력합니다.
- \(x \leq 0\)이면 함수는 \(0.1x\)을 출력합니다.

이 함수는 모델의 출력이 [0, 1] 범위 내에 있는지 확인하고 분류 오류에 비해 분석 오류에 더 많은 가중치를 부여하는 데 사용됩니다.

### 변수 목록

- \(x\): 함수의 입력 값입니다. 양수이거나 양수가 아닐 수 있습니다.

### 결론

방정식 \(\phi(x)\)는 \(x\)가 양수이면 \(x\)을 출력하고, 그렇지 않으면 \(0.1x\)을 출력하는 구간 함수입니다. 이 함수는 모델의 출력이 [0, 1] 범위 내에 있는지 확인하고 분류 오류보다 분석 오류에 더 큰 가중치를 부여하는 데 사용됩니다.

최종 답은 다음과 같습니다.
\[
\boxed{\phi(x) = \begin{cases} 
x, & \text{if } x > 0\\
0.1x, & \text{otherwise}
\end{cases}}
\]

\section*{198–246행 / env [multiline]}
assistant
주어진 LaTeX 방정식을 설명하려면 방정식을 단계별로 분석하고 각 구성 요소를 이해해야 합니다. 방정식은 다음과 같습니다.

⟦수학0⟧

### 설명

1. **\(\lambda_\textbf{coord}\) 및 \(\lambda_\textrm{no objects}\)**:
- 이는 모델의 여러 구성 요소에 대한 손실 함수의 가중치를 제어하는 ​​초매개변수입니다.

2. **합계**:
- 이 방정식은 인덱스 \(i\)와 \(j\)에 대한 합을 포함합니다. 여기서 \(i\)는 0에서 \(S^2\)까지이고 \(j\)는 0에서 \(B\)까지입니다. 여기서 \(S\)와 \(B\)는 일반적으로 각각 이미지의 높이와 너비입니다.

3. **\(\mathlarger{\mathbbm{1}}_{ij}^{\text{ obj}}\)**:
- \((i, j)\) 위치의 경계 상자가 "objects"이면 1, 그렇지 않으면 0인 지표 함수입니다.

4. **\(\mathlarger{\mathbbm{1}}_{ij}^{\text{no objects}}\)**:
- \((i, j)\) 위치의 경계 상자에 "객체 없음"이 있으면 1이고, 그렇지 않으면 0인 지표 함수입니다.

5. **\(\mathlarger{\mathbbm{1}}_i^{\text{ objects}}\)**:
- \(i\) 위치의 경계 상자가 "objects"이면 1, 그렇지 않으면 0인 지표 함수입니다.

6. **\(\mathlarger{\mathbbm{1}}_{ij}^{\text{ objects}}\)**:
- \((i, j)\) 위치의 경계 상자가 "objects"이면 1, 그렇지 않으면 0인 지표 함수입니다.

### 방정식 설명

이 방정식은 여러 항으로 구성된 손실 함수입니다.

1. **\(\lambda_\textbf{coord}\) 항**:
- 이 항은 예측된 경계 상자와 실제 경계 상자의 차이에 대한 페널티를 추가하여 모델을 정규화합니다. 두 부분의 합입니다.
- 첫 번째 부분은 예측된 경계 상자와 실제 경계 상자의 좌표 \(x\)와 \(y\) 사이의 제곱 차이의 모든 경계 상자에 대한 합입니다.
- 두 번째 부분은 예측된 경계 상자와 실제 경계 상자의 좌표 \(w\)와 \(h\) (width and height) 사이의 제곱 차이의 모든 경계 상자에 대한 합입니다.

2. **\(\lambda_\textrm{no objects}\) 항**:
- 이 항은 객체가 없을 때 예측된 경계 상자와 실제 경계 상자의 차이에 페널티를 추가하여 모델을 정규화합니다. 예측된 경계 상자와 실제 경계 상자의 차이 제곱을 모든 경계 상자에 대해 합한 값입니다.

3. **\(\sum_{i = 0}^{S^2} \sum_{j = 0}^{B} \mathlarger{\mathbbm{1}}_{ij}^{\text{ objects}} \left( p_i(c) - \hat{p}_i(c) \right)^2\) 항**:
- 이 항은 객체를 포함하는 각 경계 상자에 대해 예측된 클래스 확률과 실제 클래스 확률의 차이에 대한 페널티를 추가하여 모델을 정규화합니다. 이는 객체를 포함하는 모든 경계 상자에 대해 예측된 클래스 확률과 실제 클래스 확률의 차이 제곱을 더한 값입니다.

### 결론

이 방정식은 예측된 바운딩 박스와 그 클래스 확률을 고려하여 모델을 정규화하는 손실을 계산합니다. 손실 함수는 모델이 바운딩 박스와 그 클래스를 정확하게 예측하는 동시에 과적합을 방지하기 위해 모델을 정규화하는 여러 항으로 구성됩니다. 최종 손실은 이러한 항들의 합입니다.

⟦수학0⟧

\end{문서}
