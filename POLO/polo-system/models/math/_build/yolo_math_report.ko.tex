\\documentclass[11pt]{article}
\\usepackage[margin=1in]{geometry}
\\usepackage{amsmath, amssymb, amsfonts}
\\usepackage{hyperref}
\\usepackage{kotex}
\\setlength{\\parskip}{6pt}
\\setlength{\\parindent}{0pt}
\\title{LaTeX 수식 설명 보고서 (Middle-School Level+)}
\\author{자동 파이프라인}
\\date{2025-09-19}
\\begin{document}
\\maketitle
\\tableofcontents
\\newpage

\\section*{문서 개요}
보조원
원래 문서와 동일하게 유지되는 명확하고 간결한 기술 문서를 작성해 주세요.

\새 페이지

\section*{93–95행 / env [align]}
assistant
### 설명
이 방정식에는 두 개의 변수 \( Y_n \)와 \( \overline{Y}_n \)가 있으며, 이 변수들은 다른 변수 \( X_n \)와 \( \overline{X}_n \), 그리고 \( Z_n \)와 \( \overline{Z}_n \)를 기준으로 정의됩니다.

첫 번째 방정식은 \( Y_n \)가 \( X_n \)에 \( Z_n \)를 더한 값과 같다는 것을 나타냅니다. 즉, \( Y_n \)의 값은 \( X_n \)와 \( Z_n \)의 값의 합입니다.

두 번째 방정식은 \( \overline{Y}_n \)가 \( \overline{X}_n \)에 \( \overline{Z}_n \)를 더한 값과 같다는 것을 나타냅니다. 즉, \( \overline{Y}_n \)의 값은 \( \overline{X}_n \)와 \( \overline{Z}_n \)의 값의 합입니다.

### 결론
이 방정식의 목적은 변수 \( Y_n \)와 \( \overline{Y}_n \) 사이의 관계를 변수 \( X_n \), \( Z_n \), \( \overline{X}_n \), \( \overline{Z}_n \)를 이용하여 표현하는 것입니다. \( Y_n \)와 \( \overline{Y}_n \)를 이렇게 정의하면, 이 변수들을 이용하여 시스템의 다른 변수들 간의 관계를 분석하고 이해할 수 있습니다.

⟦수학0⟧

\section*{96–96행 / 인라인(\$ \$) }
보조원
### 설명

이 방정식에서 \(X_n\)와 \(Y_n\)는 수열입니다. 아래 첨자 \(n\)는 이 수열이 자연수 \(n\)를 색인으로 삼고 있음을 나타냅니다. 즉, 각 자연수 \(n\)에 대응하는 수열 쌍 \((X_n, Y_n)\)가 존재합니다.

수열 \(X_n\)와 \(Y_n\)는 평면 위 점의 좌표, 급수의 항, 또는 여러 점에서의 함수값 등 다양한 수학적 개념을 나타낼 수 있습니다. 정확한 해석은 방정식이 사용되는 맥락에 따라 달라집니다.

### 결론

이 방정식의 결론은 방정식이 사용되는 특정 문제나 맥락에 따라 달라지므로 명확하게 주어지지 않습니다. 그러나 제공된 정보를 바탕으로, 수열 \(X_n\)와 \(Y_n\)은 각 자연수 \(n\)에 대해 정의되어 있다고 할 수 있습니다.

⟦수학0⟧

\section*{96–96행 / 인라인(\$ \$) }
보조원
### 설명

주어진 방정식은 \(\ w\_{X\_n}\)와 \(\ w\_{Y\_n}\)입니다. 이 방정식들은 두 개의 서로 다른 변수 \(X\_n\)와 \(Y\_n\)를 나타내며, 각 변수의 가중치는 \(w\)로 표시됩니다.

간단히 말해서, \(X\_n\)와 \(Y\_n\)는 \(n\)의 값에 따라 달라지는 변수이고, \(w\)는 이 변수들과 연관된 가중치입니다. 이 변수들의 정확한 특성과 \(n\) 및 \(w\)와의 관계를 완전히 이해하려면 추가적인 맥락이나 정보가 필요합니다.

### 결론

주어진 방정식의 결론은 \(X\_n\)와 \(Y\_n\)의 가중치가 \(w\)로 표현된다는 것입니다. 다시 말해, 두 변수 \(X\_n\)와 \(Y\_n\)의 가중치는 \(w\)입니다.

⟦수학0⟧

\section*{97–99행 / env [align]}
보조자
### 설명

주어진 방정식은 \(X_n\)와 \(\ w\ X_n\)로 표시된 값의 시퀀스를 나타냅니다. \(X_n\)의 값은 \(W\)와 벡터 \(\ y^{n-1}\)라는 두 개의 입력을 받는 함수 \(\varphi_n\)에 의해 결정됩니다. 벡터 \(\ y^{n-1}\)는 \((n-1)\)번째 위치까지의 \(Y\)의 이전 값을 포함합니다. 마찬가지로, \(\ w\ X_n\)의 값은 동일한 함수 \(\varphi_n\)에 의해 결정되지만, 입력 벡터 \(Y^n\)는 \(n\) 위치에 있는 \(Y\)의 현재 값을 포함합니다.

더 간단히 말해서, 함수 \(\varphi_n\)는 시스템 (represented by \(W\) and the previous state of \(Y\))의 현재 상태를 받아서 \(X_n\)에 대한 새로운 상태를 출력합니다. 같은 함수를 사용하여 \(\ w\ X_n\)에 대한 새로운 상태를 출력하지만, \(Y\)의 현재 상태를 사용합니다.

### 결론

\(X_n\)와 \(\ w\ X_n\) 값의 시퀀스는 \(\varphi_n\) 함수를 \(Y\)와 \(W\)의 해당 상태에 반복적으로 적용하여 생성됩니다. 이 시퀀스는 시스템의 이전 상태와 외부 변수 \(W\)의 현재 상태를 기반으로 시스템의 미래 상태를 모델링하거나 예측하는 데 사용할 수 있습니다.

⟦수학0⟧

\section*{102~106행 / env [align]}
보조자
### 설명

주어진 방정식은 다음과 같습니다.

⟦수학0⟧
⟦수학1⟧

방정식의 각 부분을 단계별로 분석해 보겠습니다.

1. **합 표기법 이해하기:**
- \(\sum_{n=1}^N\)는 \(n=1\)부터 \(n=N\)까지의 항을 더한다는 것을 의미합니다.

2. **기대값:**
- \(\mathbb{E}(X_n^2)\)는 확률 변수 \(X_n\)의 제곱에 대한 기대값을 나타냅니다.
- \(\mathbb{E}(\overline{X}_n^2)\)는 표본 평균 \(\overline{X}_n\)의 제곱에 대한 기대값을 나타냅니다.

3. **부등식:**
- 부등식 \(\leq\)는 기대값의 합이 항의 개수 \(N\)와 상수 \(P\)와 \(\overline{P}\)의 곱보다 작거나 같음을 나타냅니다.

4. **상수:**
- \(P\)와 \(\overline{P}\)는 \(n\) 또는 \(N\)의 함수가 아닌 상수입니다.

### 결론

방정식의 결론은 다음과 같습니다.

⟦수학0⟧
⟦수학1⟧

이러한 결론은 확률 변수 \(X_n\)의 제곱에 대한 기대값의 합이 상수 \(P\)에 곱한 값보다 작거나 같고, 표본 평균 \(\overline{X}_n\)의 제곱에 대한 기대값의 합이 상수 \(N\)에 곱한 값보다 작거나 같다는 것을 알려줍니다.

⟦수학0⟧

\section*{107–107행 / 인라인(\$ \$) }
assistant
### 설명
\(\ w \) 방정식은 \( X_n \)와 같습니다.

### 결론
\( w \)의 값은 \( X_n \)입니다.

⟦수학0⟧

\section*{107–107행 / 인라인(\$ \$) }
보조원
### 설명

방정식 \(\snr\dfn\frac{P}{\sigma^2}\)는 모집단의 표준 편차를 나타내며, \(\snr\)로 표시합니다. 여기서 \(P\)는 모집단 평균이고 \(\sigma^2\)는 모집단의 분산입니다. 표준 편차는 값 집합의 변동 또는 분산 정도를 측정하는 척도입니다. 표준 편차가 낮으면 값이 평균에 가까운 경향이 있고, 표준 편차가 높으면 값이 더 넓은 범위에 퍼져 있음을 나타냅니다. 표준 편차는 분산의 제곱근으로 계산됩니다. 이 방정식에서 \(\snr\)는 모집단 평균 \(P\)와 분산 \(\sigma^2\)의 비율로 정의됩니다.

### 결론

모집단의 표준편차는 \(\boxed{\sqrt{\frac{P}{\sigma^2}}}\)입니다.

\section*{108–108행 / 인라인(\$ \$) }
보조원
### 설명

주어진 방정식은 \(\frac{\wh{P}}{\wh{\sigma}^2}\)입니다.

이 방정식에서:
- \(\wh{P}\)는 사건의 "거듭제곱" 또는 "확률"을 나타냅니다.
- \(\wh{\sigma}^2\)는 분포의 "분산" 또는 "산포"를 나타냅니다.

분수 \(\frac{\wh{P}}{\wh{\sigma}^2}\)는 검정력 또는 확률을 분산 또는 산포도에 대한 비율로 계산합니다. 이 비율은 산포도 또는 분산에 대한 검정력 또는 확률의 상대적 크기를 측정하는 데 사용할 수 있습니다.

좀 더 간단히 말하면, 어떤 사건의 "중요성"이나 "중요도 수준" (represented by \(\wh{P}\))을 그 사건이 얼마나 퍼져 있거나 변동성이 큰지 (represented by \(\wh{\sigma}^2\))와 비교하는 것과 같습니다.

### 결론

이 방정식의 결론은 비율 \(\frac{\wh{P}}{\wh{\sigma}^2}\)가 사건의 확산이나 분산에 비해 상대적인 중요도나 유의도를 측정한다는 것입니다.

⟦수학0⟧

\section*{108–108행 / 인라인(\$ \$) }
보조원
### 설명

주어진 방정식은 \(\dsnr\dfn\frac{\bsnr}{\snr}\)입니다.

여기서 \(\bsnr\)는 분자를, \(\snr\)는 분모를 나타냅니다. 분수 \(\frac{\bsnr}{\snr}\)는 \(\bsnr\)를 \(\snr\)로 나눈 결과입니다.

### 결론

방정식의 결론은 \(nr\)의 값이 분수 \(\frac{\bsnr}{\snr}\)와 같다는 것입니다.

⟦수학0⟧

\section*{109–109행 / 인라인(\$ \$) }
보조원
### 설명

\( R \dfn \frac{\log{M}}{N} \) 방정식은 분자가 \( M \)의 로그이고 분모가 \( N \)인 비율을 나타냅니다. 단계별 분석은 다음과 같습니다.

1. **\( M \)의 로그**: ​​로그 함수는 \( \log \)로 표기되며, 지수 함수의 역함수입니다. 이 함수는 \( M \)를 얻기 위해 로그 (usually 10 or \( e \))의 밑을 몇 제곱해야 하는지 알려줍니다. 이 경우에는 \( M \)의 로그를 취합니다.

2. **\( N \)로 나누기**: \( M \)의 로그 값을 \( N \)로 나눕니다. 이 나눗셈은 로그 값을 \( N \) 인수로 조정합니다.

3. **표기법 \( R \dfn \)**: 기호 \( \dfn \)는 종종 새로운 변수나 함수를 정의하는 데 사용됩니다. 여기서 \( R \)는 \( M \)의 로그를 \( N \)로 나눈 결과로 정의됩니다.

### 결론

이 방정식의 결론은 단순히 \( R \)의 값이며, 이는 \( M \)의 로그를 \( N \)로 나눈 결과입니다. 따라서 결론은 다음과 같이 나타낼 수 있습니다.

⟦수학0⟧

즉, \( R \)는 \( M \)의 로그 값과 \( N \)의 로그 값의 비율입니다.

\박스형{R = \frac{\log{M}}{N}}

\end{문서}
