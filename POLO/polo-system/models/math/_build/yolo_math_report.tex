\\documentclass[11pt]{article}
\\usepackage[margin=1in]{geometry}
\\usepackage{amsmath, amssymb, amsfonts}
\\usepackage{hyperref}
\\usepackage{kotex}
\\setlength{\\parskip}{6pt}
\\setlength{\\parindent}{0pt}
\\title{LaTeX Equation Explanation Report (Middle-School Level+)}
\\author{Automatic Pipeline}
\\date{2025-09-19}
\\begin{document}
\\maketitle
\\tableofcontents
\\newpage

\\section*{Document Overview}
assistant
You are a clear, concise technical write-up that remaining the same as the original write-up.

\newpage

\section*{Lines 93–95 / env [align]}
assistant
### Explanation
In this equation, we have two variables \( Y_n \) and \( \overline{Y}_n \), which are defined in terms of other variables \( X_n \) and \( \overline{X}_n \), and \( Z_n \) and \( \overline{Z}_n \).

The first equation states that \( Y_n \) is equal to \( X_n \) plus \( Z_n \). This means that the value of \( Y_n \) is the sum of the values of \( X_n \) and \( Z_n \).

The second equation states that \( \overline{Y}_n \) is equal to \( \overline{X}_n \) plus \( \overline{Z}_n \). This means that the value of \( \overline{Y}_n \) is the sum of the values of \( \overline{X}_n \) and \( \overline{Z}_n \).

### Conclusion
The purpose of this equation is to express the relationship between the variables \( Y_n \) and \( \overline{Y}_n \) in terms of the variables \( X_n \), \( Z_n \), \( \overline{X}_n \), and \( \overline{Z}_n \). By defining \( Y_n \) and \( \overline{Y}_n \) in this way, we can use these variables to analyze and understand the relationships between the other variables in the system.

\[
\boxed{Y_n = X_n + Z_n \quad \text{and} \quad \overline{Y}_n = \overline{X}_n + \overline{Z}_n}
\]


\section*{Lines 96–96 / inline(\$ \$) }
assistant
### Explanation

In this equation, \(X_n\) and \(Y_n\) are sequences of numbers. The subscript \(n\) indicates that these sequences are indexed by the natural number \(n\). This means that for each natural number \(n\), there is a corresponding pair of numbers \((X_n, Y_n)\).

The sequences \(X_n\) and \(Y_n\) could represent various mathematical concepts, such as the coordinates of points in a plane, the terms of a series, or the values of a function at different points. The exact interpretation depends on the context in which the equation is used.

### Conclusion

The conclusion of this equation is not explicitly given, as it depends on the specific problem or context in which the equation is used. However, based on the information provided, we can say that the sequences \(X_n\) and \(Y_n\) are defined for each natural number \(n\).

\[
\boxed{\text{The sequences } X_n \text{ and } Y_n \text{ are defined for each natural number } n.}
\]


\section*{Lines 96–96 / inline(\$ \$) }
assistant
### Explanation

The given equations are \(\ w\_{X\_n}\) and \(\ w\_{Y\_n}\). These equations represent two different variables, \(X\_n\) and \(Y\_n\), each with a weight represented by the letter \(w\).

In simpler terms, \(X\_n\) and \(Y\_n\) are variables that depend on the value of \(n\), and \(w\) is a weight associated with these variables. The exact nature of these variables and their relationship to \(n\) and \(w\) would require additional context or information to fully understand.

### Conclusion

The conclusion of the given equations is that the weight of \(X\_n\) and \(Y\_n\) is represented by \(w\). In other words, the weight of both variables \(X\_n\) and \(Y\_n\) is \(w\).

\[
\boxed{w}
\]


\section*{Lines 97–99 / env [align]}
assistant
### Explanation

The given equation represents a sequence of values denoted by \(X_n\) and \(\ w\ X_n\). The value of \(X_n\) is determined by a function \(\varphi_n\) which takes two inputs: \(W\) and a vector \(\ y^{n-1}\). The vector \(\ y^{n-1}\) contains the previous values of \(Y\) up to the \((n-1)\)-th position. Similarly, the value of \(\ w\ X_n\) is determined by the same function \(\varphi_n\) but with the input vector \(Y^n\), which includes the current value of \(Y\) at position \(n\).

In simpler terms, the function \(\varphi_n\) takes the current state of the system (represented by \(W\) and the previous state of \(Y\)) and outputs a new state for \(X_n\). The same function is used to output a new state for \(\ w\ X_n\) but with the current state of \(Y\).

### Conclusion

The sequence of values \(X_n\) and \(\ w\ X_n\) is generated by repeatedly applying the function \(\varphi_n\) to the appropriate states of \(Y\) and \(W\). The sequence can be used to model or predict future states of a system based on its previous states and the current state of an external variable \(W\).

\[
\boxed{X_n = \varphi_n(W, \ y^{n-1}), \quad \ w\ X_n = \ w\ \varphi_n(Y^n)}
\]


\section*{Lines 102–106 / env [align]}
assistant
### Explanation

The given equations are:

\[
\sum_{n=1}^N\mathbb{E}(X_n^2) \leq N\cdot P 
\]
\[
\sum_{n=1}^N\mathbb{E}(\overline{X}_n^2) \leq N\cdot \overline{P}
\]

Let's break down each part of the equations step by step.

1. **Understanding the Summation Notation:**
   - \(\sum_{n=1}^N\) means we are summing up the terms from \(n=1\) to \(n=N\).

2. **Expected Value:**
   - \(\mathbb{E}(X_n^2)\) represents the expected value of the square of the random variable \(X_n\).
   - \(\mathbb{E}(\overline{X}_n^2)\) represents the expected value of the square of the sample mean \(\overline{X}_n\).

3. **Inequality:**
   - The inequalities \(\leq\) indicate that the sums of the expected values are less than or equal to the products of the number of terms \(N\) and some constants \(P\) and \(\overline{P}\).

4. ** constants:**
   - \(P\) and \(\overline{P}\) are constants that are not functions of \(n\) or \(N\).

### Conclusion

The conclusions of the equations are:

\[
\sum_{n=1}^N\mathbb{E}(X_n^2) \leq N P
\]
\[
\sum_{n=1}^N\mathbb{E}(\overline{X}_n^2) \leq N \overline{P}
\]

These conclusions tell us that the sum of the expected values of the squares of the random variables \(X_n\) is less than or equal to \(N\) times the constant \(P\), and the sum of the expected values of the squares of the sample means \(\overline{X}_n\) is less than or equal to \(N\) times the constant \(\overline{P}\).

\(\boxed{\sum_{n=1}^N\mathbb{E}(X_n^2) \leq N P, \quad \sum_{n=1}^N\mathbb{E}(\overline{X}_n^2) \leq N \overline{P}}\)


\section*{Lines 107–107 / inline(\$ \$) }
assistant
### Explanation
The equation \(\ w \) is equal to \( X_n \).

### Conclusion
The value of \( w \) is \( X_n \).

\[
\boxed{w = X_n}
\]


\section*{Lines 107–107 / inline(\$ \$) }
assistant
### Explanation

The equation \(\snr\dfn\frac{P}{\sigma^2}\) represents the standard deviation of a population, denoted by \(\snr\). Here, \(P\) is the population mean, and \(\sigma^2\) is the variance of the population. The standard deviation is a measure of the amount of variation or dispersion in a set of values. A low standard deviation indicates that the values tend to be close to the mean, while a high standard deviation indicates that the values are spread out over a wider range. The standard deviation is calculated as the square root of the variance. In this equation, \(\snr\) is defined as the ratio of the population mean \(P\) to the variance \(\sigma^2\).

### Conclusion

The standard deviation of the population is \(\boxed{\sqrt{\frac{P}{\sigma^2}}}\).


\section*{Lines 108–108 / inline(\$ \$) }
assistant
### Explanation

The given equation is \(\frac{\wh{P}}{\wh{\sigma}^2}\).

In this equation:
- \(\wh{P}\) represents the "power" or "probability" of an event.
- \(\wh{\sigma}^2\) represents the "variance" or " spread" of a distribution.

The fraction \(\frac{\wh{P}}{\wh{\sigma}^2}\) calculates the ratio of the power or probability to the variance or spread. This ratio can be used to measure the relative magnitude of the power or probability compared to the spread or variance.

In simpler terms, it's like comparing the "importance" or "importance level" of an event (represented by \(\wh{P}\)) to how spread out or variable that event is (represented by \(\wh{\sigma}^2\)).

### Conclusion

The conclusion of this equation is that the ratio \(\frac{\wh{P}}{\wh{\sigma}^2}\) provides a measure of the relative importance or significance of an event compared to its spread or variance.

\(\boxed{\frac{\wh{P}}{\wh{\sigma}^2}}\)


\section*{Lines 108–108 / inline(\$ \$) }
assistant
### Explanation

The given equation is \(\dsnr\dfn\frac{\bsnr}{\snr}\).

Here, \(\bsnr\) represents the numerator and \(\snr\) represents the denominator of the fraction. The fraction \(\frac{\bsnr}{\snr}\) is the result of dividing \(\bsnr\) by \(\snr\).

### Conclusion

The conclusion of the equation is that the value of \(nr\) is equal to the fraction \(\frac{\bsnr}{\snr}\).

\(\boxed{nr = \frac{\bsnr}{\snr}}\)


\section*{Lines 109–109 / inline(\$ \$) }
assistant
### Explanation

The equation \( R \dfn \frac{\log{M}}{N} \) represents a ratio where the numerator is the logarithm of \( M \) and the denominator is \( N \). Here's a step-by-step breakdown:

1. **Logarithm of \( M \)**: The logarithm function, denoted as \( \log \), is the inverse of the exponential function. It tells us to what power we need to raise the base of the logarithm (usually 10 or \( e \)) to get \( M \). In this case, we are taking the logarithm of \( M \).

2. **Divide by \( N \)**: The result of the logarithm of \( M \) is then divided by \( N \). This division scales the logarithmic value by the factor \( N \).

3. **Notation \( R \dfn \)**: The symbol \( \dfn \) is often used to define a new variable or function. Here, \( R \) is defined as the result of the division of the logarithm of \( M \) by \( N \).

### Conclusion

The conclusion of this equation is simply the value of \( R \), which is the result of the division of the logarithm of \( M \) by \( N \). Therefore, the conclusion can be stated as:

\[ R = \frac{\log{M}}{N} \]

This means that \( R \) is the ratio of the logarithm of \( M \) to \( N \).

\boxed{R = \frac{\log{M}}{N}}


\end{document}
